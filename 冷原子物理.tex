\documentclass[a4paper,12pt]{article}
\usepackage{ctex}

% 字体设置
\usepackage{amsmath}   % 数学符号
\allowdisplaybreaks  % 允许跨页显示多行公式
\usepackage{amssymb}   % 数学符号
\usepackage{amsfonts}  % 数学字体
\usepackage{graphicx}  % 插入图片
\usepackage{fancyhdr}  % 页眉页脚设置
\usepackage{geometry}  % 页面布局
\usepackage[colorlinks=true, linkcolor=black, citecolor=black, urlcolor=black]{hyperref}  % 超链接支持,设置链接为黑色
\usepackage{listings}  % 程序代码显示
\usepackage{tikz}      % 绘图
\usepackage{titlesec}  % 标题设置
%\usepackage{fontspec}  % 字体支持(适用于XeLaTeX/LuaLaTeX)

%%%%%%%%%%%%%%%%%%%%%%
% 方框相关
\usepackage{tcolorbox}  % 创建方框
% 加载列表库,支持多页方框
\tcbuselibrary{listingsutf8}
\tcbuselibrary{breakable} % 确保启用breakable库
% \begin{tcolorbox}[colback=red!3, colframe=red!60!black, title=注释, breakable]
% 	Gauss's Law 
% \end{tcolorbox}
%%%%%%%%%%%%%%%%%%%%%%

%%%%%%%%%%%%%%%%%%%%%%
% 设置斜体
\usepackage{CJKfntef}
% \CJKunderdot{\textit{能标分离}}
%%%%%%%%%%%%%%%%%%%%%%


% 额外设置
\numberwithin{equation}{section}  % 设置公式按section编号
\numberwithin{figure}{section}  % 设置图表按节编号

% 页面布局
\geometry{left=2.5cm, right=2.5cm, top=3cm, bottom=3cm}

% 封面设置
\title{
    \vspace{-3cm}
    \Huge{\textbf{原子分子光物理}} \\
    \vspace{0.5cm}
    \Huge{冷原子物理} \\
    \vspace{1.5cm}
    \Large{讲义编号: 001} \\
    \vspace{2cm}
    \large{译者: 李亚宸} \\
    \vfill
    \Large{日期: \today}
}

\author{} % 作者为空,封面不用显示作者
\date{}

% 页眉页脚设置
\pagestyle{fancy}
\fancyhf{}
\fancyhead[L]{冷原子物理}
\fancyhead[C]{}
\fancyhead[R]{\today}
\fancyfoot[C]{\thepage}

% 章节格式
\titleformat{\section}[block]{\normalfont\Large\bfseries}{\thesection}{1em}{}
\titleformat{\subsection}[runin]{\normalfont\large\bfseries}{\thesubsection}{1em}{}
\titleformat{\subsubsection}[runin]{\normalfont\normalsize\bfseries}{\thesubsubsection}{1em}{}

\begin{document}

% 封面
\maketitle
\newpage

% 目录
\tableofcontents
\newpage

% 第一章
\section{A Single Atom}\label{sec:A Single Atom}
\subsection{Electronic Structure}\label{sec:Electronic Structure}

首先,我们考虑原子核周围 $Z$ 个电子运动的一般的哈密顿量,其中包含库仑相互作用、自旋轨道耦合和超精细耦合。这三个效应共同决定了原子的电子结构。在此需要强调\CJKunderdot{\textit{能标分离}}(the separation of energy scales)的重要性:这三项的能量量级存在显著差异。得益于能标分离,我们可以逐项分析这些效应,从而获得电子结构的清晰图像。

\textbf{Coulomb Interaction between Electron and Nucleus. }每个电子围绕原子核运动,与原子核之间存在库仑吸引作用,该相互作用由下式给出
\begin{equation}\label{eq:1.1}
	\hat{H}_0=\sum_{i=1}^Z\left(-\frac{\hbar^2\nabla_i^2}{2m^*}+V_{\mathrm{ei}}(\mathbf{r}_i)\right)
\end{equation}
其中 \( i = 1, \dots, Z \) 标记电子,\( \mathbf{r}_i \) 是电子相对于原子核的位置坐标,\( m^* = mM/(m + M) \) 是约化质量,其中 \( m \) 是电子质量,\( M \) 是原子核质量;\( V_{\text{ei}}(\mathbf{r}) = -Z\kappa/r \) 是电子与原子核之间的库仑势,其中 \( \kappa = e^2/(4\pi \varepsilon_0) \),\( e \) 是电子电荷,\( \varepsilon_0 \) 是真空介电常数。该体系的本征态由三个量子数 \( (n, l, m) \) 来表征。通常,对于球对称势能,由于 \( SO(3) \) 旋转对称性,能量仅依赖于 \( n \) 和 \( l \),而与 \( m \) 无关。然而,对于 \( 1/r \) 的库仑势能,如氢原子,能量为:
\begin{equation}\label{eq:1.2}
	E=-\frac{m^*Z^2\kappa^2}{2\hbar^2n^2}
\end{equation}
能级仅依赖于主量子数 \( n \),并且与角量子数 \( l \) 无关;其中 \( l \) 可以取从 0 到 \( n-1 \) 的整数值。这种额外的简并性是势能为 \( 1/r \) 的结果,它导致了比三维旋转对称性更大的 \( SO(4) \) 对称性。这些能级的间隔大约为电子伏特级别(约 \( 10^{14} \) Hz),因为它源自于库仑相互作用。通常将这些能级画在“能级图”上,如图 \ref{fig:1.1} 所示,其中 1, 2, 3, ... 代表主量子数 \( n \),而 \( s, p, d, ... \) 表示角量子数 \( l \)。氢原子的能级图如图 \ref{fig:1.1}(a) 所示。
\begin{figure}[h!]
	\centering
	\includegraphics[width=12cm]{images/1-1.png}  % 指定相对路径
	\caption{能级示意图:(a) 无屏蔽效应的氢原子,(b) 有屏蔽效应的碱金属原子。}
	\label{fig:1.1}
\end{figure}

\textbf{Coulomb Interaction between Electrons. }电子之间的库仑排斥作用由以下公式给出:
\begin{equation}\label{eq:1.3}
	\hat{V}_\mathrm{c}=\sum_{i<j}V_\mathrm{ee}(\mathbf{r}_i-\mathbf{r}_j)
\end{equation}
其中 \( V_{ee}(r) = \kappa/r \)。这里我们讨论这一项的几个物理结果。首先,完全填充的内层电子屏蔽了原子核的正电荷 \( Ze \),因此价电子感受到一个约化的库仑势。当最外层电子的轨道远离原子核时,大致上,它感受到的是被所有其他 \( Z-1 \) 个电子屏蔽的场。也就是说,当 \( r \) 足够大时,电子与原子核之间的有效吸引变成了一个有效正电荷为 1 的库仑势能,即 \( -\kappa/r \)。当这个电子靠近原子核时,它会更多地感受到未被屏蔽的核势能,电荷为 \( Ze \)。电子与原子核之间的库仑吸引在 \( r \) 足够小的情况下恢复为 \( -Z\kappa/r \)。因此,价电子看到的有效势能不再与 \( 1/r \) 成正比,从而更大的 \( SO(4) \) 对称性不再存在。因此,具有相同主量子数 \( n \) 但不同角量子数 \( l \) 的本征态不再简并,能级变为:
\begin{equation}\label{eq:1.4}
		E=-\frac{m^*Z^2\kappa^2}{2\hbar^2(n-\delta(n,l))^2}
\end{equation}
其中 \( \delta(n, l) \) 是一个依赖于 \( n \) 和 \( l \) 的函数,被称为“量子数亏损”。带有屏蔽效应的能级图如图 \ref{fig:1.1}(b) 所示。通常,具有较大 \( l \) 的能级较高。这个能级分裂的大小也是电子伏特量级,因为它也来源于库仑作用。

其次,考虑两个电子在两个轨道中,例如 \( \psi_1(\mathbf{r}_1) \) 和 \( \psi_2(\mathbf{r}_2) \)。由于两个电子的总波函数必须是反对称的,如果这两个电子形成一个自旋单重态,则自旋空间中的波函数是交换反对称的,那么它们的空间波函数必须是交换对称的,即 \( \psi_1(\mathbf{r}_1) \psi_2(\mathbf{r}_2) + \psi_1(\mathbf{r}_2) \psi_2(\mathbf{r}_1) \)。如果这两个电子形成自旋三重态,则自旋空间中的波函数是交换对称的,空间波函数应当是交换反对称的,即 \( \psi_1(\mathbf{r}_1) \psi_2(\mathbf{r}_2) - \psi_1(\mathbf{r}_2) \psi_2(\mathbf{r}_1) \)。在后者情况下(自旋三重态),当两个电子在空间上足够接近时,波函数为零,这使得两个电子之间的库仑排斥减小。因此,自旋三重态的能量低于单重态的能量。换句话说,库仑排斥倾向于最大化电子的总自旋 \( S \)。这个论点可以推广到更多电子的情况以及更多量子态的情况,从而给出了\CJKunderdot{\textit{洪特定则}}(Hund’s rule)第一条的早期解释。同时,对于给定的 \( S \),当总角动量 \( L \) 最大化时,短程排斥作用被最小化,这给出了洪特定则的第二条。洪特定则的特征能量量级也大致是电子伏特量级的。

\textbf{The Spin-Orbit and Hyperfine Couplings. }自旋-轨道耦合的哈密顿量由下式给出:
\begin{equation}\label{eq:1.5}
	\hat{H}_{\mathrm{so}}=\sum_i\alpha_\mathrm{f}^i\hat{\mathbf{S}}_i\cdot\hat{\mathbf{L}}_i
\end{equation}
它描述了电子自旋 \( \mathbf{\hat{S}}_i \) 和其轨道角动量 \( \mathbf{\hat{L}}_i \) 之间的耦合,耦合强度为 \( \alpha _{\text{f}}^{i} \),从而产生了精细结构。自旋-轨道耦合的来源可以通过以下直观的方式理解。处于电子的静止参考系中,原子核围绕电子运动。由于原子核带电,原子核的环形运动会产生电流,且电流的强度与电子和原子核之间相对运动的角动量成正比。这个环形电流进一步产生了磁场,磁场作用于电子的自旋。这导致了由 \eqref{eq:1.5} 式给出的自旋-轨道耦合。

从这个物理图像中可以看出,由于这个过程涉及由电流产生的磁效应,它自然比库仑相互作用要弱,因为后者是纯粹的电子效应。事实上,自旋-轨道耦合的特征能量量级通常约为 \( 10^{-3} \) eV(约 \( 10^{11} \) Hz),在许多情况下,它比源自库仑相互作用的洪特定则要弱得多。从本质上来讲,这种自旋-轨道耦合是每个电子的自旋和轨道角动量之间的耦合;然而,由于洪特定则将所有价电子的自旋锁定在总自旋 \( \mathbf{\hat{S}} \) 的本征态上,并将所有价电子的角动量锁定在总角动量 \( \mathbf{\hat{L}} \) 的本征态上,因此用 \( \mathbf{\hat{S}} \) 和 \( \mathbf{\hat{L}} \) 来表示自旋-轨道耦合的主阶效应将更加方便,即 \( \alpha_\text{f} \mathbf{\hat{S}} \cdot \mathbf{\hat{L}} + \cdots \)。这里,第一项表示自旋和轨道角动量之间的耦合,强度为 \( \alpha_\text{f} \),被称为 LS 耦合。由 \( \cdots \) 表示的剩余项表示实际耦合[\eqref{eq:1.5} 式]与 LS 耦合之间的差异,因此 \( S \) 和 \( L \) 对于 \eqref{eq:1.5} 式并不是好量子数,这些剩余项与洪特定则竞争,并且可以改变量子数 \( S \) 和 \( L \)。尽管如此,$ \mathbf{\hat{J}}=\mathbf{\hat{S}}+\mathbf{\hat{L}}=\sum_i{\mathbf{\hat{J}}_i} $ 仍然与自旋-轨道耦合[\eqref{eq:1.5} 式]对易。

超精细相互作用将电子的自由度 \( \mathbf{\hat{S}} \) 和 \( \mathbf{\hat{L}} \) 与原子核的自旋 \( \mathbf{\hat{I}} \) 耦合。一般来说,\( \mathbf{\hat{S}} \) 和 \( \mathbf{\hat{L}} \) 与 \( \mathbf{\hat{I}} \) 的耦合方式是不同的。然而,超精细耦合的特征能量量级大约是 \( 10^{-6} \) eV(约 \( 10^8 \) – \( 10^9 \) Hz),比自旋-轨道耦合要小得多。这是因为核磁子远小于玻尔磁子。由于 LS 耦合已经将 \( \mathbf{\hat{S}} \) 和 \( \mathbf{\hat{L}} \) 锁定在 \( \mathbf{\hat{J}} \) 的本征态上,因此我们用 \( \mathbf{\hat{J}} \) 和 \( \mathbf{\hat{I}} \) 来表示超精细耦合的主导阶效应,即 \( \alpha_\text{hf} \mathbf{\hat{J}} \cdot \mathbf{\hat{I}} + \cdots \),其中 \( \alpha_\text{hf} \) 是该耦合的强度。这会产生超精细结构。只有第一项时,\( J \) 仍然是一个好的量子数,但由 \( \cdots \) 表示的残余项可以改变量子数 \( J \),这是因为 \( \mathbf{\hat{S}} \) 和 \( \mathbf{\hat{L}} \) 与 \( \mathbf{\hat{I}} \) 的耦合方式不同。

\subsection{Magnetic Structure}\label{sec:Magnetic Structure}
现在我们考虑静磁场对原子结构的影响。由于电子带电,原则上,原子内的电子运动也会受到磁场的影响。然而,与库仑相互作用相比,这种影响非常小,因此我们可以放心地忽略磁场对电子轨道的改变。于是我们只关注作用于电子自旋 \( \mathbf{\hat{S}} \)、轨道角动量 \( \mathbf{\hat{L}} \) 和核自旋 \( \mathbf{\hat{I}} \) 的塞曼效应。对于实验室中数百高斯的磁场,\CJKunderdot{\textit{塞曼分裂}}(Zeeman splitting)的能量量级与\CJKunderdot{\textit{超精细分裂}}(hyperfine splitting)相当。

因此,在这里我们将原子视为一个携带 \( \mathbf{\hat{S}} \)、\( \mathbf{\hat{L}} \) 和 \( \mathbf{\hat{I}} \) 的中性质点。现在让我们聚焦于碱金属原子的基态。例如,对于 \( ^{87}\text{Rb} \) 原子,具有 \( S = 1/2 \)、\( L = 0 \) 和 \( I = 3/2 \),基态自旋结构由下式决定:
\begin{equation}\label{eq:1.6}
	\hat{H}_\mathrm{s}=B(\mu_\mathrm{B}g_\mathrm{S}\hat{S}_z+\mu_\mathrm{N}g_\mathrm{I}\hat{I}_z)+\alpha_\mathrm{hf}\hat{\mathbf{J}}\cdot\hat{\mathbf{I}}
\end{equation}

\begin{tcolorbox}[colback=red!3, colframe=red!60!black, title=注释, breakable]
	由于 $ L=0 $,所以
	\begin{align}
		\hat{H}_{\text{S}}&=\left( \boldsymbol{\mu }_{\text{S}}\cdot \boldsymbol{B}+\boldsymbol{\mu }_{\text{I}}\cdot \boldsymbol{B} \right) +\alpha _{\text{hf}}\mathbf{\hat{J}}\cdot \mathbf{\hat{I}} \notag \\
		&=\left( g_{\text{S}}\mu _{\text{B}}\hat{S}_z+g_{\text{I}}\mu _{\text{N}}\hat{I}_z \right) B+\alpha _{\text{hf}}\mathbf{\hat{J}}\cdot \mathbf{\hat{I}} \notag \\
		&=B\left( \mu _{\text{B}}g_{\text{S}}\hat{S}_z+\mu _{\text{N}}g_{\text{I}}\hat{I}_z \right) +\alpha _{\text{hf}}\mathbf{\hat{J}}\cdot \mathbf{\hat{I}} \notag
	\end{align}
	得到式 \ref{eq:1.6}。
\end{tcolorbox}

其中,\( \mu_\text{B} \) 和 \( \mu_\text{N} \) 分别是玻尔磁子和核磁子,并且 \( \mu_\text{N} \ll \mu_\text{B} \)。\( g_S \) 和 \( g_I \) 是朗德 \( g \)-因子。这里我们首先考虑磁场在空间上是均匀的,且其方向选择为 \( \hat{z} \) 方向。对于这个哈密顿量,\( F_z \) 是一个\CJKunderdot{\textit{好量子数}}(good quantum number),其能量本征值可以精确求解。在这里,我们分别分析小 \( B \) 和大 \( B \) 区域的行为。通过平滑地连接小 \( B \) 和大 \( B \) 区域,自然得到能级的定性特征,如图 \ref{fig:1.3} 所示。

\begin{figure}[h!]
	\centering
	\includegraphics[width=14cm]{images/1-3.png}  % 指定相对路径
	\caption{塞曼能级结构示意图。这里我们考虑 \( ^{87}\text{Rb} \) 原子的电子基态,具有 \( J = 1/2 \) 和 \( I = 3/2 \)。\( F_z \) 标记了每个态的好量子数。}
	\label{fig:1.3}
\end{figure}

在 \( B \) 很小的区域,当 \( B\mu_\text{B} g_s \ll \alpha_\text{hf} \) 时,超精细耦合占主导地位。超精细耦合将能级分裂成 \( F = 1 \) 的态和 \( F = 2 \) 的态。在 \( F = 1 \) 的三重态或 \( F = 2 \) 的五重态内,\CJKunderdot{\textit{塞曼磁场}}(Zeeman field)仅产生线性塞曼能量和二阶塞曼能能量,分别与 \( F_z \) 和 \( F_z^2 \) 成正比。存在二阶塞曼效应的原因是因为\CJKunderdot{\textit{超精细自旋}}(hyperfine spin)包含了电子自旋和核自旋两部分,它们与外磁场的耦合方式不同。

在 \( B \) 很大的区域,当 \( B\mu_\text{B} g_S \gg \alpha_\text{hf} \) 时,电子自旋的塞曼能量占主导地位。四个 \( S_z \approx -1/2 \) 的态的能量随着 \( B \) 增加而减小,而另外四个 \( S_z \approx 1/2 \) 的态的能量随着 \( B \) 增加而增大。

\textbf{Magnetic Trap. }在存在磁场的情况下,一些自旋态的能量随着磁场的增加而上升。也就是说,如果一个原子被制备在这样的态中,它可以被困在磁场为局部极小值的区域中。处于这些态的原子被称为\CJKunderdot{\textit{低场寻态}}(low-field seeking)原子。而另一些态的能量随着磁场的增加而下降,这些态可以被困在磁场为局部极大值的区域中。处于这些态的原子被称为\CJKunderdot{\textit{高场寻态}}(high-field seeking)原子。这就是磁阱的基本原理。不难证明,由于麦克斯韦方程的限制,在真空腔中如果没有电流存在,磁场强度不可能具有局部极大值。也可以证明,为了在空间中产生磁场的局部极小值,磁场的方向不可能处处相同。因此,一个自然的想法是利用空间变化的磁场,在磁场强度的极小值处俘获低场寻态原子。例如,所谓的四极阱具有的磁场构型为 \( \mathbf{B} = B_0(x, y, -2z) \),其中 \( \mathbf{r} = (x, y, z) \) 是空间坐标,磁场强度在 \( \mathbf{r} = 0 \) 处具有极小值。

\textbf{Emergent Synthetic Gauge Field. }上面我们研究了原子在均匀磁场中的塞曼能量,这使我们产生了在磁场的极小值附近俘获原子的想法。另一方面,我们也注意到,磁场不可能指向相同的方向以在其强度上产生局部极小值。因此,为了填补这一空白,我们必须考虑原子在空间变化的磁场 \( \mathbf{B}(\mathbf{r}) \) 中的运动。考虑描述运动原子的薛定谔方程:
\begin{equation}\label{eq:1.7}
	i\hbar\frac{\partial\psi}{\partial t}=\left(-\frac{\hbar^2}{2m}\nabla^2+\hat{H}_\mathrm{s}(\mathbf{r})\right)\psi
\end{equation}
其中
\begin{equation}\label{eq:1.8}
	\hat{H}_\mathrm{s}(\mathbf{r})=\mu_\mathrm{B}g_\mathrm{S}\mathbf{B}(\mathbf{r})\cdot\hat{\mathbf{S}}+\mu_\mathrm{N}g_\mathrm{I}\mathbf{B}(\mathbf{r})\cdot\hat{\mathbf{I}}+\alpha_\mathrm{hf}\mathbf{\hat{J}}\cdot\hat{\mathbf{I}}
\end{equation}

我们引入一个幺正变换 \( \mathcal{U}(\mathbf{r}) \) 来对 \( \hat{H}_s(\mathbf{r}) \) 进行对角化,即 \( \Lambda \left( \mathbf{r} \right) = \mathcal{U}^\dagger(\mathbf{r}) \hat{H}_s(\mathbf{r}) \mathcal{U}(\mathbf{r}) \) 对于每个 \( \mathbf{r} \) 都成立。对于每个 \( \mathbf{r} \),我们总是可以选择一个局域坐标系,使得磁场方向沿局域坐标系的 \( \hat{z} \) 方向,因此,\( \hat{H}_s(\mathbf{r}) \) 在该局域坐标下具有与图 \ref{fig:1.3} 中所示相同的能级结构,哈密顿量 \ref{eq:1.6} 只依赖于 \( |\mathbf{B}(\mathbf{r})| \)。

令 \( \tilde{\psi} = U^\dagger(\mathbf{r}) \psi \),则 \( \tilde{\psi} \) 的薛定谔方程可以写为:
\begin{equation}\label{eq:1.9}
	i\hbar\frac{\partial\tilde{\psi}}{\partial t}=\left(\frac{1}{2m}(-i\hbar\nabla-\mathbf{A})^2+\Lambda(\mathbf{r})\right)\tilde{\psi}
\end{equation}
其中 \( \mathbf{A}(\mathbf{r}) = i\hbar \mathcal{U}^\dagger(\mathbf{r}) (\nabla \mathcal{U}(\mathbf{r})) \)。这里我们用到了:
\begin{equation}\label{eq:1.10}
	\mathcal{U}^\dagger(-i\hbar\nabla)\mathcal{U}=-i\hbar\nabla-i\hbar\mathcal{U}^\dagger(\mathbf{r})(\nabla\mathcal{U}(\mathbf{r}))
\end{equation}
注意到 \( \mathcal{U}^\dagger(\mathbf{r}) \mathcal{U}(\mathbf{r}) = 1 \),这意味着:
\begin{equation}\label{eq:1.11}
	\mathcal{U}^\dagger(\mathbf{r})(\nabla\mathcal{U}(\mathbf{r}))+(\nabla\mathcal{U}^\dagger(\mathbf{r}))\mathcal{U}(\mathbf{r})=0
\end{equation}
于是 \( \mathcal{U}^\dagger(\mathbf{r}) (\nabla \mathcal{U}(\mathbf{r})) \) 是纯虚数,因此 \( \mathbf{A} \) 是一个实数的场。式 \ref{eq:1.9} 形式上与粒子在规范场中运动的薛定谔方程相同。在冷原子文献中,这一涌现的规范场 \( \mathbf{A} \) 被称为\CJKunderdot{\textit{人工规范场}}(synthetic gauge field)。

\begin{tcolorbox}[colback=red!3, colframe=red!60!black, title=注释, breakable]
	式 \ref{eq:1.7} 的两边作用上幺正变换有
	\begin{align}
		i\hbar\frac{\partial \psi}{\partial t}&=\left( -\frac{\hbar^2}{2m}\nabla ^2+\hat{H}_{\text{s}}\left( \mathbf{r} \right) \right) \psi  \notag \\
		i\hbar\mathcal{U}^{\dag}\frac{\partial \psi}{\partial t}&=\mathcal{U}^{\dag}\left( -\frac{\hbar^2}{2m}\nabla ^2+\hat{H}_{\text{s}}\left( \mathbf{r} \right) \right) \mathcal{U}\mathcal{U}^{\dag}\psi  \notag \\
		i\hbar\frac{\partial}{\partial t}\mathcal{U}^{\dag}\psi &=\left( \frac{1}{2m}\mathcal{U}^{\dag}\left( -i\hbar\nabla \right) ^2\mathcal{U}+\mathcal{U}^{\dag}\hat{H}_{\text{s}}\left( \mathbf{r} \right) \mathcal{U} \right) \mathcal{U}^{\dag}\psi \tag{1}
	\end{align}
	考虑到式 \ref{eq:1.10},式 1 右边第二项为
	\begin{align}
		\mathcal{U}^{\dag}\left( -i\hbar\nabla \right) ^2\mathcal{U}&=\left[ \mathcal{U}^{\dag}\left( -i\hbar\nabla \right) \mathcal{U} \right] \left[ \mathcal{U}^{\dag}\left( -i\hbar\nabla \right) \mathcal{U} \right]  \notag \\
		&=\left[ -i\hbar\mathcal{U}^{\dag}\left( \nabla \mathcal{U} \right) -i\hbar\nabla \right] \left[ -i\hbar\mathcal{U}^{\dag}\left( \nabla \mathcal{U} \right) -i\hbar\nabla \right]  \notag \\
		&=\left( -i\hbar\nabla -\mathbf{A} \right)^2 \tag{2}
	\end{align}
	因此变换后的薛定谔方程为
	\begin{align}
		i\hbar\frac{\partial}{\partial t}\mathcal{U}^{\dag}\psi &=\left( \frac{1}{2m}\left( -i\hbar\nabla -\mathbf{A} \right) ^2+\mathcal{U}^{\dag}\hat{H}_{\text{s}}\left( \mathbf{r} \right) \mathcal{U} \right) \mathcal{U}^{\dag}\psi  \notag \\
		i\hbar\frac{\partial}{\partial t}\tilde{\psi}&=\left( \frac{1}{2m}\left( -i\hbar\nabla -\mathbf{A} \right) ^2+\Lambda \right) \tilde{\psi}  \tag{3}
	\end{align}
	上式即为式 \ref{eq:1.9}。
\end{tcolorbox}

规范场分为阿贝尔和非阿贝尔两类。对于阿贝尔规范场,规范场的不同分量彼此对易,例如当 \( \mathbf{A} \) 的三个分量都只是一个实数时。对于非阿贝尔规范场,规范场的不同分量彼此不对易。我们前面定义的 \( \mathbf{A} \) 一般来说是一个矩阵,且 \( \mathbf{A} \) 的不同空间分量不对易,这代表了一个非阿贝尔规范场。\( \mathbf{A} \) 的非对角元引起了局域坐标中不同自旋本征态之间的跃迁。如果这些非对角元与 \( \Lambda \left( \mathbf{r} \right) \) 的对角元之间的差异相比,过于微小,则跃迁效应变得可以忽略,此时可以放心地假设原子始终处于同一绝热自旋本征态中,这通常发生在磁场强度较大且空间变化较小的情况下。在\CJKunderdot{\textit{绝热近似}}(adiabatic approximation)下,处于这种绝热自旋态的原子可以有效地被视为一个无自旋粒子,它仅能感受到依赖于 \( |\mathbf{B}(\mathbf{r})| \) 的势能。唯一的修正是规范场 \( \mathbf{A} \) 的对角元,记作 \( A_{ii} \),对应于由 \( i \) 标记的绝热自旋本征态。这产生了一个阿贝尔规范场。当它导致一个非零的人工磁场 \( \mathbf{B}_\text{syn} \) 时,这个阿贝尔规范场会产生物理效应,人工磁场由 \( \mathbf{B}_\text{syn} = \nabla \times \mathbf{A}_{ii}(\mathbf{r}) \) 给出。

\begin{figure}[h!]
	\centering
	\includegraphics[width=12cm]{images/1-4.png}  % 指定相对路径
	\caption{人工规范场。(a) 一个中性且具有自旋的原子(用实心圆表示)在一个磁场中运动,该磁场的方向在空间中变化(用较长的箭头表示),原子的自旋方向(从实心圆中伸出的箭头)始终沿局部磁场方向。(b) 一个无自旋的带电粒子在一个磁场(用竖直箭头表示)中运动。当以下两个条件满足时,这两种运动是等效的:(i) 子图 (a) 中的绝热近似成立,(ii) 任意轨迹上累积的相位在两种情况下始终相等。}
	\label{fig:1.4}
\end{figure}

上述讨论表明,在空间变化的磁场 \( \mathbf{B}(\mathbf{r}) \) 中,带自旋的中性原子的运动等效于在人工磁场 \( \mathbf{B}_\text{syn} \) 中无自旋带电粒子的运动。这个人工磁场 \( \mathbf{B}_\text{syn} \) 不应与真实磁场 \( \mathbf{B} \) 混淆。真实磁场 \( \mathbf{B} \) 仅作用于原子的自旋自由度,并不与原子的运动耦合,因为原子是电中性的。然而,人工规范场仅作用于带电粒子的运动自由度。这个等效性是由\CJKunderdot{\textit{贝里相位}}(Berry phase)效应引起的,其基本思想如图 \ref{fig:1.4} 所示。让我们考虑一个带自旋的中性原子在空间变化的磁场中运动的情况。根据绝热近似,当原子运动时,其自旋方向总是沿着局域磁场的方向。因此,当原子在空间中沿闭合轨迹运动时,它的波函数会获得一个额外的相位,这个相位正比于自旋方向沿运动轨迹所张成的立体角。另一方面,对于在磁场中运动的带电粒子,由于\CJKunderdot{\textit{阿哈罗诺夫–玻姆效应}}(Aharonov–Bohm effect),它也需要为任何闭合轨迹引入一个相位,这个相位与轨迹所围成的总磁通量成正比。如果这两个相位对于任何轨迹始终相等,那么在空间变化的磁场中,带自旋的中性原子的空间运动可以有效地用“带电”的无自旋粒子在人工磁场中的运动来描述。这也告诉我们,以这种方式产生的人工磁通量取决于磁场方向在空间中的变化速度。然而,真实磁场在空间中非常迅速地变化是困难的。在第 \ref{sec:Light Shift} 节中,我们将介绍矢量光频移,它可以为原子产生一个光诱导的塞曼场,这个场可以在粒子间距量级的空间尺度上变化。通过结合这两种效应,我们可以产生一个强的人工磁场。

\begin{tcolorbox}[colback=red!3, colframe=red!60!black, title=Box 1.1 不同种类的“磁场”, breakable]
	区分真实磁场、光诱导磁场和人工磁场这三个概念非常重要。这里我们总结了这三种“磁场”之间的差异和联系。首先,前两者作用于原子的自旋自由度,而第三个作用于原子的空间运动。其次,真实磁场和光诱导磁场均能以空间依赖的方式使原子自旋极化,从而产生人工磁场。第三,由光诱导磁场产生的人工磁场可以比由真实磁场产生的人工磁场强得多。人工规范场是冷原子物理学中的一个活跃研究课题,更多讨论可以在 Box 7.3 中找到。
\end{tcolorbox}

最后,我们还应提到,当由于磁场方向的快速变化使的 \( \mathbf{A} \) 的非对角元变得非常大,或当 \(\Lambda(\mathbf{r})\) 的对角元之间的差异变得非常小时,这种绝热近似可能会失效。例如,后者发生在四极阱的 \( \mathbf{r} = 0 \) 处,在那里一些自旋态会变得简并。在 \( \mathbf{r} = 0 \) 附近区域,\( \mathbf{A} \) 的非对角元将会起关键作用,它会驱动不同绝热自旋态之间的跃迁。因此,低场寻态有很大概率会翻转成高场寻态,该过程亦称为\CJKunderdot{\textit{马约拉纳跃迁}}(Majorana transition)。当这种跃迁发生时,原子无法被磁阱俘获。这是 1995 年之前在磁阱中实现玻色-爱因斯坦凝聚的主要挑战之一。这实际上是一个非阿贝尔规范场的效应。有趣的是,尽管人工规范场在 2010 年左右成为冷原子物理学中的一个主要研究课题,但是它的效应在这个领域诞生之前就已经存在。JILA 小组和 MIT 小组提出了不同的方法来解决这个问题。JILA 小组采用一个振荡的偏置磁场来处理这个问题。处理一个含时周期系统需要采用\CJKunderdot{\textit{弗洛凯理论}}(Floquet theory),这将在第 7.4 节中讨论。

\subsection{Light Shift}\label{sec:Light Shift}
在前两节中,我们讨论了单个原子的结构。在本节中,我们将讨论单个原子是如何与激光相互作用的。

\textbf{General Framework. }在激光场中,原子会感受到一个有效势。这里我们将以碱金属原子的基态为例讨论这种势。可以看到,这种\CJKunderdot{\textit{光诱导势}}(light-induced
potential)的前两项主阶贡献是标量势和矢量势,分别称为标量光频移和矢量光频移。

\begin{figure}[h!]
	\centering
	\includegraphics[width=4cm]{images/1-5.png}  % 指定相对路径
	\caption{原子与光相互作用的能级结构图。这里我们以碱金属原子为例。图中标出了三个有关的能标:激发能 \( E_\text{ex} \)、失谐 \( \Delta_\text{e} \) 和精细结构分裂 \( \alpha_\text{f} \)。}
	\label{fig:1.5}
\end{figure}

正如在第 \ref{sec:Electronic Structure} 节中讨论的那样,我们考虑基态 \( ^2S_{1/2} \) 系列,其中电子角动量 \( L = 0 \),以及电子激发态 \( ^2P_{1/2} \) 和 \( ^2P_{3/2} \) 系列,其中 \( L = 1 \)。这里,“\CJKunderdot{\textit{系列}}”(manifold)指同时包含电子自旋和核自旋自由度,每个系列包含多个自旋态。基态和激发态系列之间的能级间距 \( E_\text{ex} \) 大约在\CJKunderdot{\textit{电子伏特}}(electron volt)量级,通常处于可见光的能量范围内,如图 \ref{fig:1.5} 所示。我们定义\CJKunderdot{\textit{失谐}}(detuning)为 \( \Delta_\text{e} = E_\text{ex} - \hbar\omega \),其中 \( \omega \) 是激光频率。通常,\( \Delta_\text{e} \) 可与精细结构分裂相当或大于精细结构分裂,并且远大于超精细结构分裂。另外,\( \Delta_\text{e} \) 也远小于其他电子激发态的失谐,也就是说,所有其他电子激发态的都是远失谐的。基于上述能标的考量,我们建立如下的假设:

\textbullet\
$^{2}P_{1/2}$ 与 $^{2}P_{3/2}$ 的贡献均十分重要,且应在模型中予以平等对待。

\textbullet\
超精细耦合和塞曼能量可以放心地忽略。

\textbullet\
除 $^{2}S_{1/2}$、$^{2}P_{1/2}$ 和 $^{2}P_{3/2}$ 系列,其余所有电子态均未包含在模型中。

因此,我们将模型表述如下:
\begin{align}\label{eq:1.12}
	\hat{H} = \hat{H}_\text{at} + \hat{H}_\text{d}
\end{align}
$\hat{H}_{\text{at}}$ 表示原子部分的哈密顿量,其表达式为:
\begin{align}\label{eq:1.13}
	\hat{H}_\text{at} = E_\text{ex} \mathcal{P}_\text{e} + \alpha_\text{f} \hat{\mathbf{S}} \cdot \hat{\mathbf{L}}
\end{align}
这里我们定义投影到 \( L = 0 \) 的基态的投影算符 \( \mathcal{P}_g \),和投影到 \( L = 1 \) 的激发态的投影算符 \( \mathcal{P}_e \),并且满足 \( \mathcal{P}_g + \mathcal{P}_e = 1 \)。我们还将基态能量定为零,并且 \( \hat{\mathbf{S}} \cdot \hat{\mathbf{L}} \) 作用在基态系列上为零。还需要注意的是,\( \hat{H}_\text{at} \) 本身并不提供基态和激发态之间的耦合。

这里的原子-光相互作用是指原子内部的电子与激光的电磁场之间的相互作用。考虑对电子施加一个时变电场,哈密顿量可以写为
\begin{align}\label{eq:1.14}
	\hat{H} = \sum_{i=1}^{Z} \frac{1}{2m^*} (\hat{\mathbf{p}}_i + e \mathbf{A}(t))^2 + \ldots
\end{align}
其中 \( \mathbf{r}_i \) 表示在核静止坐标系中价电子的位置,省略号(. . .)代表第 \ref{sec:Electronic Structure} 节中讨论的库仑相互作用和其他项。通常,激光的波长远大于原子的大小,因此,在原子内部电子波函数的空间尺度上,我们可以忽略 \( \mathbf{A} \) 的空间依赖性,只考虑它的时间依赖性。正如我们将在下面看到的,这会诱导出\CJKunderdot{\textit{偶极耦合}}(dipolar coupling)。

这里我们进行规范变换,考虑 \( \hat{\mathcal{U}}^\dagger \hat{H} \hat{\mathcal{U}} \rightarrow \hat{H} \),以及 \( \hat{\mathcal{U}}^\dagger \psi \rightarrow \psi \),其中
\begin{align}\label{eq:1.15}
	\hat{\mathcal{U}} = e^{-\sum_{i} i e \mathbf{A} \cdot \mathbf{r}_i/\hbar}
\end{align}
在这个规范变换下,\(\hat{\mathbf{p}}_i + e\mathbf{A}\) 变回 \(\hat{\mathbf{p}}_i\),动能变回 \(\hat{\mathbf{p}}_i^2 / (2m^*)\)。然而,由于这个规范变换是时间依赖的,哈密顿量中会增加一个额外的项 \(i\hbar (\partial_t \hat{\mathcal{U}})^\dagger \hat{\mathcal{U}}\),其表达式为
\begin{align}\label{eq:1.16}
	i \hbar (\partial_t \hat{\mathcal{U}}^\dagger) \hat{\mathcal{U}} = -e \mathbf{r}_i \cdot \frac{\partial \mathbf{A}}{\partial t} = e \mathbf{r}_i \cdot \mathbf{E}
\end{align}
其中 \( \mathbf{E} \) 是由 \( \mathbf{E} = -\partial \mathbf{A}/\partial t \) 给出的光场的电场部分。因此,这个额外项就是偶极耦合 \(\hat{H}_\text{d} = \mathbf{d} \cdot \mathbf{E}\),其中 \(\mathbf{d} = \sum_i e\mathbf{r}_i\) 是电偶极算符,它能引起基态和激发态之间的跃迁。

\begin{tcolorbox}[colback=red!3, colframe=red!60!black, title=注释, breakable]
	由式 \ref{eq:1.15} 有
	\begin{align}
		\frac{\partial \mathcal{U}^{\dag}}{\partial t}=\mathcal{U}^{\dag}\sum_i{i}e\frac{\partial \mathbf{A}}{\partial t}\cdot \mathbf{r}_i/\hbar\tag{1}
	\end{align}
	在薛定谔方程的两边作用上幺正变换
	\begin{align}
		&i\hbar\frac{\partial \psi}{\partial t}=\hat{H}\psi  \notag \\
		&i\hbar\mathcal{U}^{\dag}\frac{\partial \psi}{\partial t}=\mathcal{U}^{\dag}\hat{H}\mathcal{U}\mathcal{U}^{\dag}\psi  \notag \\ 
		&i\hbar\frac{\partial}{\partial t}\left( \mathcal{U}^{\dag}\psi \right) =\mathcal{U}^{\dag}\hat{H}\mathcal{U}\left( \mathcal{U}^{\dag}\psi \right) +i\hbar\left( \frac{\partial \mathcal{U}^{\dag}}{\partial t}\mathcal{U} \right) \left( \mathcal{U}^{\dag}\psi \right)  \notag \\ 
		&i\hbar\frac{\partial}{\partial t}\left( \mathcal{U}^{\dag}\psi \right) =\left[ \mathcal{U}^{\dag}\hat{H}\mathcal{U}+i\hbar\left( \frac{\partial \mathcal{U}^{\dag}}{\partial t}\mathcal{U} \right) \right] \left( \mathcal{U}^{\dag}\psi \right)  \tag{2} \\ 
		&i\hbar\frac{\partial}{\partial t}\left( \mathcal{U}^{\dag}\psi \right) =\left[ \mathcal{U}^{\dag}\hat{H}\mathcal{U}+i\hbar\left( \sum_i{i}e\frac{\partial \mathbf{A}}{\partial t}\cdot \mathbf{r}_i/\hbar \right) \right] \left( \mathcal{U}^{\dag}\psi \right)  \notag \\ 
		&i\hbar\frac{\partial}{\partial t}\left( \mathcal{U}^{\dag}\psi \right) =\left( \mathcal{U}^{\dag}\hat{H}\mathcal{U}-\sum_i{e\frac{\partial \mathbf{A}}{\partial t}\cdot \mathbf{r}_i} \right) \mathcal{U}^{\dag}\psi  \tag{3}
	\end{align}
	根据 Hausdorff 公式
	\begin{align}
		e^ABe^{-A}=B+\left[ A,B \right] +\frac{1}{2!}\left[ A,\left[ A,B \right] \right] +\cdot \cdot \cdot \tag{4}
	\end{align}
	有
	\begin{align}
		&\left( e^{+ie\mathbf{A}\cdot \mathbf{r}_i/\hbar} \right) \mathbf{\hat{p}}_i\left( e^{-ie\mathbf{A}\cdot \mathbf{r}_i/\hbar} \right) \notag \\
		&=\left( e^{+ie\left( A_xx+A_yy+A_zz \right) /\hbar} \right) \left( \hat{p}_x\boldsymbol{e}_x+\hat{p}_y\boldsymbol{e}_y+\hat{p}_z\boldsymbol{e}_z \right) \left( e^{-ie\left( A_xx+A_yy+A_zz \right) /\hbar} \right)  \notag \\
		&=\left( \hat{p}_x-eA_x \right) \boldsymbol{e}_x+\left( \hat{p}_y-eA_y \right) \boldsymbol{e}_y+\left( \hat{p}_z-eA_z \right) \boldsymbol{e}_z \notag \\
		&=\left( \mathbf{\hat{p}}_i-e\mathbf{A} \right) \tag{5}
	\end{align}
	所以
	\begin{align}
		\mathcal{U}^{\dag}\left( \mathbf{\hat{p}}_i+e\mathbf{A} \right) ^2\mathcal{U}&=\left[ \mathcal{U}^{\dag}\left( \mathbf{\hat{p}}_i+e\mathbf{A} \right) \mathcal{U} \right] \left[ \mathcal{U}^{\dag}\left( \mathbf{\hat{p}}_i+e\mathbf{A} \right) \mathcal{U} \right]  \notag \\
		&=\left[ \left( e^{+\sum_i{i}e\mathbf{A}\cdot \mathbf{r}_i/\hbar} \right) \left( \mathbf{\hat{p}}_i+e\mathbf{A} \right) \left( e^{-\sum_i{i}e\mathbf{A}\cdot \mathbf{r}_i/\hbar} \right) \right] ^2 \notag \\
		&=\left[ \left( e^{+ie\mathbf{A}\cdot \mathbf{r}_i/\hbar} \right) \mathbf{\hat{p}}_i\left( e^{-ie\mathbf{A}\cdot \mathbf{r}_i/\hbar} \right) +e\mathbf{A} \right] ^2 \notag \\
		&=\left[ \left( \mathbf{\hat{p}}_i-e\mathbf{A} \right) +e\mathbf{A} \right] ^2 \notag \\
		&=\mathbf{\hat{p}}_{i}^{2}  \tag{6}
	\end{align}
\end{tcolorbox}

首先,我们考虑一个单一激光场,并且
\begin{align}\label{eq:1.17}
	\hat{H}_\text{d} = \mathbf{d} \cdot \mathbf{E} = \sum_{j=x,y,z} d_j E_j^0 \cos(\phi_j - \omega t)
\end{align}
其中 \( E_j^0 \cos(\phi_j - \omega t) \) 是电场的第 \( j \) 个分量,\( j = x, y, z \) 表示光场的三个空间分量。例如,在这种表示法中,对于沿 \(\hat{x}\) 方向线性偏振的光,\( E_x^0 \neq 0 \) 且 \( E_y^0 = E_z^0 = 0 \);对于沿 \((\hat{x} + \hat{y})/\sqrt{2}\) 方向线性偏振的光,\( E_x^0 = E_y^0 \neq 0 \),\( \phi_x = \phi_y \) 且 \( E_z^0 = 0 \);对于在 \( xy \) 平面内圆偏振的光,\( E_x^0 = E_y^0 \neq 0 \),\( \phi_x = \phi_y \pm \pi/2 \) 且 \( E_z^0 = 0 \)。

这里我们通过\CJKunderdot{\textit{旋转波近似}}(rotating wave approximation)来化简哈密顿量。我们定义一个幺正变换
\begin{align}\label{eq:1.18}
	\hat{\mathcal{U}}(t) = e^{-i \omega t \mathcal{P}_e} = (1 - \mathcal{P}_e) + \mathcal{P}_e e^{-i \omega t} = \mathcal{P}_g + \mathcal{P}_e e^{-i \omega t}
\end{align}
\begin{tcolorbox}[colback=red!3, colframe=red!60!black, title=注释, breakable]
	\begin{align}
		\hat{\mathcal{U}}\left( t \right) &=e^{-i\omega t\mathcal{P}_e} \notag \\
		&=1-i\omega t\mathcal{P}_e+\frac{1}{2}\left( -i\omega t \right) ^2\mathcal{P}_{e}^{2}+\frac{1}{3!}\left( -i\omega t \right) ^3\mathcal{P}_{e}^{3}+... \notag \\
		&=1-i\omega t\mathcal{P}_e+\frac{1}{2}\left( -i\omega t \right) ^2\mathcal{P}_e+\frac{1}{3!}\left( -i\omega t \right) ^3\mathcal{P}_e+... \notag \\
		&=\left( 1-\mathcal{P}_e \right) +\mathcal{P}_e-i\omega t\mathcal{P}_e+\frac{1}{2}\left( -i\omega t \right) ^2\mathcal{P}_e+\frac{1}{3!}\left( -i\omega t \right) ^3\mathcal{P}_e+... \notag \\
		&=\left( 1-\mathcal{P}_e \right) +\mathcal{P}_e\left[ 1-i\omega t+\frac{1}{2}\left( -i\omega t \right) ^2+\frac{1}{3!}\left( -i\omega t \right) ^3+... \right]  \notag \\
		&=\left( 1-\mathcal{P}_e \right) +\mathcal{P}_ee^{-i\omega t} \tag{1}
	\end{align}
\end{tcolorbox}

将这个幺正变换作用到哈密顿量上。旋转波近似忽略了 \( e^{\pm i 2\omega t} \) 项,因为它们是高频振荡项。采用这种近似,可以得出做变换后的哈密顿量为
\begin{align}\label{eq:1.19}
	\hat{H}_\text{d} = \hat{\mathcal{U}}^\dagger(t) \hat{H}_\text{d} \hat{\mathcal{U}}(t) \approx \frac{1}{2} \sum_{j=x,y,z} \left( \mathcal{E}_j^* \mathcal{P}_g d_j \mathcal{P}_e + \mathcal{E}_j \mathcal{P}_e d_j \mathcal{P}_g \right)
\end{align}

\begin{tcolorbox}[colback=red!3, colframe=red!60!black, title=注释, breakable]
	考虑到
	\begin{align}
		\left( \mathcal{P}_g+\mathcal{P}_ee^{+i\omega t} \right) \left( \mathcal{P}_g+\mathcal{P}_ee^{-i\omega t} \right) &=\mathcal{P}_g+\mathcal{P}_g\mathcal{P}_ee^{-i\omega t}+\mathcal{P}_e\mathcal{P}_ge^{+i\omega t}+\mathcal{P}_e \notag \\
		&=\mathcal{P}_g+\mathcal{P}_e=1 \tag{1}
	\end{align}
	将式 \ref{eq:1.18} 作用到偶极耦合上
	\begin{align}
		\mathcal{U}^{\dag}\hat{H}_{\text{d}}\mathcal{U}&=\left( \mathcal{P}_g+\mathcal{P}_ee^{+i\omega t} \right) \hat{H}_{\text{d}}\left( \mathcal{P}_g+\mathcal{P}_ee^{-i\omega t} \right)  \notag \\
		&=\left( \mathcal{P}_g+\mathcal{P}_ee^{+i\omega t} \right) \left[ \sum_j{d_jE_{j}^{0}\cos \left( \phi _j-\omega t \right)} \right] \left( \mathcal{P}_g+\mathcal{P}_ee^{-i\omega t} \right)  \notag \\
		&=\left( \mathcal{P}_g+\mathcal{P}_ee^{+i\omega t} \right) \left[ \frac{1}{2}\sum_j{d_jE_{j}^{0}\left( e^{i\left( \phi _j-\omega t \right)}+e^{-i\left( \phi _j-\omega t \right)} \right)} \right] \left( \mathcal{P}_g+\mathcal{P}_ee^{-i\omega t} \right)  \notag \\
		&=\frac{1}{2}\sum_j{E_{j}^{0}\left( \mathcal{P}_g+\mathcal{P}_ee^{+i\omega t} \right) d_j\left( e^{i\left( \phi _j-\omega t \right)}+e^{-i\left( \phi _j-\omega t \right)} \right) \left( \mathcal{P}_g+\mathcal{P}_ee^{-i\omega t} \right)} \notag \\
		&=\frac{1}{2}\sum_j{E_{j}^{0}\left[ \mathcal{P}_gd_j\mathcal{P}_e\left( e^{i\left( \phi _j-2\omega t \right)}+e^{-i\phi _j} \right) +\mathcal{P}_ed_j\mathcal{P}_g\left( e^{i\phi _j}+e^{-i\left( \phi _j-2\omega t \right)} \right) \right]} \notag \\
		&\approx \frac{1}{2}\sum_j{E_{j}^{0}\left[ \mathcal{P}_gd_j\mathcal{P}_ee^{-i\phi _j}+\mathcal{P}_ed_j\mathcal{P}_ge^{i\phi _j} \right]} \notag \\
		&=\frac{1}{2}\sum_j{\left( \mathcal{E}_{j}^{*}\mathcal{P}_gd_j\mathcal{P}_e+\mathcal{E}_j\mathcal{P}_ed_j\mathcal{P}_g \right)}  \tag{2}
	\end{align}
	得到式 \ref{eq:1.19}。
\end{tcolorbox}

其中我们用到了 \( P_g d_j P_g = P_e d_j P_e = 0 \),这是由电子波函数的旋转对称性导致的。这里 \( \mathcal{E}_j \) 定义为 \( E_{j}^{0} e^{i\phi_j} \)。与 \( \mathbf{E} \) 不同,这里的 \( \mathcal{E} \) 是时间无关的,它包含了光场的相位和振幅信息。在这种记号中,对于沿 \(\hat{x}\) 方向线性偏振的光,\( \mathbf{\mathcal{E}}=E_0\mathbf{e}_{\mathbf{x}} \);对于沿 \((\hat{x} + \hat{y})/\sqrt{2}\) 方向线性偏振的光,\( \mathcal{E}=\frac{E_0}{\sqrt{2}}\left( \mathbf{e}_{\mathbf{x}}+\mathbf{e}_{\mathbf{y}} \right)  \);对于在 \( xy \) 平面内圆偏振的光,\( \mathcal{E}=\frac{E_0}{\sqrt{2}}\left( \mathbf{e}_{\mathbf{x}}+i\mathbf{e}_{\mathbf{y}} \right)  \)。

此外,由幺正变换产生的额外时间依赖项 \( i\hbar (\partial_t \hat{U}^\dagger) \hat{U} \) 可以被吸收到 \(\hat{H}_{\text{at}}\) 中,使得 \(\hat{H}_{\text{at}}\) 变为
\begin{align}\label{eq:1.20}
	\hat{H}_\text{at} = \Delta_\text{e} \mathcal{P}_\text{e} + \alpha_\text{f} \hat{\mathbf{S}} \cdot \hat{\mathbf{L}}
\end{align}
其中 \(\Delta_\text{e} = E_\text{ex} - \hbar \omega\)。\(\Delta_\text{e} > 0\) 称为\CJKunderdot{\textit{红失谐}}(red detuning),\(\Delta_\text{e} < 0\) 称为\CJKunderdot{\textit{蓝失谐}}(blue detuning)。这样,我们得到新的 \(\hat{H}_{\text{at}} + \hat{H}_\text{d}\) 是一个时间无关的哈密顿量。我们将在第 7.4 节中用 Floquet 理论重新讨论这样的周期性问题。

\begin{tcolorbox}[colback=red!3, colframe=red!60!black, title=注释, breakable]
	由
	\begin{align}
		\hat{\mathcal{U}}\left( t \right) &=\mathcal{P}_g+\mathcal{P}_ee^{-i\omega t} \tag{1}
	\end{align}
	可得
	\begin{align}
		i\hbar\left( \frac{\partial}{\partial t}\mathcal{U}^{\dag} \right) \mathcal{U}&=i\hbar\left[ \frac{\partial}{\partial t}\left( \mathcal{P}_g+\mathcal{P}_ee^{+i\omega t} \right) \right] \left( \mathcal{P}_g+\mathcal{P}_ee^{-i\omega t} \right) \notag \\
		&=i\hbar\left( +i\omega \mathcal{P}_ee^{+i\omega t} \right) \left( \mathcal{P}_g+\mathcal{P}_ee^{-i\omega t} \right) \notag \\ 
		&=\left( -\hbar\omega \right) \left( \mathcal{P}_ee^{+i\omega t} \right) \left( \mathcal{P}_g+\mathcal{P}_ee^{-i\omega t} \right)  \notag \\
		&=-\hbar\omega \left( \mathcal{P}_e\mathcal{P}_ge^{+i\omega t}+\mathcal{P}_e\mathcal{P}_e \right)  \notag \\
		&=-\hbar\omega \mathcal{P}_e  \tag{2}
	\end{align}
	又由于
	\begin{align}
		\mathcal{U}\mathcal{P}_{\text{e}}&=\left( \mathcal{P}_{\text{g}}+\mathcal{P}_{\text{e}}e^{-i\omega t} \right) \mathcal{P}_{\text{e}}=\mathcal{P}_{\text{e}}e^{-i\omega t} \notag \\
		\mathcal{P}_{\text{e}}\mathcal{U}&=\mathcal{P}_{\text{e}}\left( \mathcal{P}_{\text{g}}+\mathcal{P}_{\text{e}}e^{-i\omega t} \right) =\mathcal{P}_{\text{e}}e^{-i\omega t} \notag \\
		\Rightarrow \mathcal{U}\mathcal{P}_{\text{e}}&=\mathcal{P}_{\text{e}}\mathcal{U}   \tag{3} \\
		\Delta_\text{e} &= E_\text{ex} - \hbar\omega \tag{4}
	\end{align}
	所以做幺正变换后的完整哈密顿量为
	\begin{align}
		\mathcal{U}^{\dag}\hat{H}\mathcal{U}+i\hbar\left( \frac{\partial}{\partial t}\mathcal{U}^{\dag} \right) \mathcal{U}&=\mathcal{U}^{\dag}\left( \hat{H}_{\text{at}}+\hat{H}_{\text{d}} \right) \mathcal{U}+i\hbar\left( \frac{\partial}{\partial t}\mathcal{U}^{\dag} \right) \mathcal{U} \notag \\
		&=\mathcal{U}^{\dag}\left[ \left( E_{\text{ex}}\mathcal{P}_{\text{e}}+\alpha _{\text{f}}\mathbf{\hat{S}}\cdot \mathbf{\hat{L}} \right) +\hat{H}_{\text{d}} \right] \mathcal{U}+i\hbar\left( \frac{\partial}{\partial t}\mathcal{U}^{\dag} \right) \mathcal{U} \notag \\
		&=\left( E_{\text{ex}}\mathcal{P}_{\text{e}}+\alpha _{\text{f}}\mathbf{\hat{S}}\cdot \mathbf{\hat{L}} \right) +\mathcal{U}^{\dag}\hat{H}_{\text{d}}\mathcal{U}-\hbar\omega \mathcal{P}_{\text{e}} \notag \\
		&=\left( \Delta _{\text{e}}\mathcal{P}_{\text{e}}+\alpha _{\text{f}}\mathbf{\hat{S}}\cdot \mathbf{\hat{L}} \right) +\left( \mathcal{U}^{\dag}\hat{H}_{\text{d}}\mathcal{U} \right)  \notag \\
		&=\hat{H}_{\text{at}}+\hat{H}_{\text{d}}  \tag{5}
	\end{align}
\end{tcolorbox}

这种方法可以直接推广到多束激光的情况。在这种情况下,电场包括了所有激光束的贡献,我们将偶极耦合写为
\begin{align}\label{eq:1.21}
	\hat{H}_\text{d} = \mathbf{d} \cdot \mathbf{E} = \sum_{\kappa} \sum_{j=x,y,z} d_j E_{j,\kappa}^0 \cos(\phi_{j,\kappa} - \omega_\kappa t)
\end{align}
其中 \( \kappa \) 标记不同的光场。这里不同的激光可以有不同的频率,但它们的差异应该很小。也就是说,假设我们将这些频率的平均值记为 \( \omega \);它们之间的差异应该远小于 \( \omega \)。因此,我们仍然可以放心地舍弃所有带有 \( e^{\pm i(\omega_\kappa + \omega)t} \) 的项,并保留所有带有 \( e^{\pm i(\omega_\kappa - \omega)t} \) 的项。经过旋转波近似后,很容易证明式 \ref{eq:1.19} 仍然成立,只是 \( \mathcal{E}_j \) 的定义应修改为
\begin{align}\label{eq:1.22}
	\mathcal{E}_j = \sum_{\kappa} E_{j,\kappa}^0 e^{i \phi_{j,\kappa} - i(\omega_\kappa - \omega) t}
\end{align}
如果所有激光具有相同的频率,那么 \( \mathcal{E} \) 是与时间无关的。如果存在多个频率,\( \mathcal{E} \) 是与时间有关的。

旋转波近似的物理意义可以通过将激光视为由光子数标记的量子场来更清楚地理解。现在让我们考虑一个处于基态(表示为 \(|g\rangle\))的原子在一个具有 \(N\) 个光子的光场(表示为 \(|N\rangle\))中。在二次量子化的表述下,\(\mathbf{d} \cdot \mathbf{E}\) 可以在将原子从基态耦合到激发态时产生一个光子或湮灭一个光子。因此,我们需要考虑以下两个不同的二阶微扰过程。 以 \(|g\rangle|N\rangle\) 为初态,原子可以通过吸收一个光子激发到激发态(表示为 \(|e\rangle\)),使得中间态为 \(|e\rangle|N-1\rangle\),能量为 \(\Delta_e = E_{\text{ex}} - \hbar \omega \),然后通过发射一个光子回到基态 \(|g\rangle|N\rangle\)。原子也可以通过发射一个光子激发到激发态,使得中间态为 \(|e\rangle|N+1\rangle\),能量为 \(E_{\text{ex}} + \hbar \omega\),然后通过吸收一个光子回到基态 \(|g\rangle|N\rangle\)。第二个过程具有更高的中间态能量,因此,其概率要小得多。旋转波近似就是忽略第二个过程。仅考虑第一个过程时,算上光子的能量,失谐从式 \ref{eq:1.13} 中的 \(E_{\text{ex}}\) 变为式 \ref{eq:1.20} 中的 \(\Delta_e\)。

下面我们将考虑在将激发态 \(|e\rangle|N-1\rangle\) 积掉后,这个二阶过程如何为处于基态的原子生成一个有效势。根据微扰理论,基态原子的有效哈密顿量为
\begin{align}\label{eq:1.23}
	\hat{H}_{\text{eff}} = - \mathcal{P}_g \hat{H}_\text{d} \mathcal{P}_e \hat{H}_\text{at}^{-1} \mathcal{P}_e \hat{H}_\text{d} \mathcal{P}_g = - \frac{1}{4} \sum_{i,j=x,y,z} \mathcal{E}_i^* \hat{\mathcal{D}}_{ij} \mathcal{E}_j
\end{align}
其中 \( \hat{\mathcal{D}}_{ij} \) 是一个二阶笛卡尔张量算符,定义为
\begin{align}\label{eq:1.24}
	\hat{\mathcal{D}}_{ij} = \mathcal{P}_g d_i \mathcal{P}_e \hat{H}_\text{at}^{-1} \mathcal{P}_e d_j \mathcal{P}_g
\end{align}

\begin{tcolorbox}[colback=red!3, colframe=red!60!black, title=注释, breakable]
	将式 \ref{eq:1.19} 带入式 \ref{eq:1.23} 左边
	\begin{align}
		\hat{H}_{\text{eff}}&=-\mathcal{P}_g\hat{H}_{\text{d}}\mathcal{P}_e\hat{H}_{\text{at}}^{-1}\mathcal{P}_e\hat{H}_{\text{d}}\mathcal{P}_g \notag \\
		&=-\frac{1}{4}\mathcal{P}_g\left( \sum_i{\mathcal{E}_{i}^{*}\mathcal{P}_gd_i\mathcal{P}_e+\mathcal{E}_i\mathcal{P}_ed_i\mathcal{P}_g} \right) \mathcal{P}_e\hat{H}_{\text{at}}^{-1}\mathcal{P}_e\left( \sum_j{\mathcal{E}_{j}^{*}\mathcal{P}_gd_j\mathcal{P}_e+\mathcal{E}_j\mathcal{P}_ed_j\mathcal{P}_g} \right) \mathcal{P}_g \notag \\
		&=-\frac{1}{4}\sum_{i,j}{\mathcal{P}_g\left( \mathcal{E}_{i}^{*}\mathcal{P}_gd_i\mathcal{P}_e+\mathcal{E}_i\mathcal{P}_ed_i\mathcal{P}_g \right) \mathcal{P}_e\hat{H}_{\text{at}}^{-1}\mathcal{P}_e\left( \mathcal{E}_{j}^{*}\mathcal{P}_gd_j\mathcal{P}_e+\mathcal{E}_j\mathcal{P}_ed_j\mathcal{P}_g \right) \mathcal{P}_g} \notag \\
		&=-\frac{1}{4}\sum_{i,j}{\left( \mathcal{E}_{i}^{*}\mathcal{P}_gd_i\mathcal{P}_e \right) \hat{H}_{\text{at}}^{-1}\left( \mathcal{E}_j\mathcal{P}_ed_j\mathcal{P}_g \right)} \notag \\
		&=-\frac{1}{4}\sum_{i,j}{\mathcal{E}_{i}^{*}\left( \mathcal{P}_gd_i\mathcal{P}_e\hat{H}_{\text{at}}^{-1}\mathcal{P}_ed_j\mathcal{P}_g \right)}\mathcal{E}_j \tag{1}
	\end{align}
	得到式 \ref{eq:1.23} 右边
\end{tcolorbox}

\(\hat{\mathcal{D}}_{ij}\) 完全是由原子决定的。利用 \(\mathcal{P}_g + \mathcal{P}_e = 1\) 和 \(\mathcal{P}_g d_i \mathcal{P}_g = 0\),可以消去式 \ref{eq:1.24} 中的 \(\mathcal{P}_e\)。下面我们考虑不同情况下 \(\hat{\mathcal{D}}_{ij}\) 的性质。

\begin{tcolorbox}[colback=red!3, colframe=red!60!black, title=注释, breakable]
	\begin{align}
		\hat{\mathcal{D}}_{ij}&=\mathcal{P}_gd_i\mathcal{P}_e\hat{H}_{\text{at}}^{-1}\mathcal{P}_ed_j\mathcal{P}_g \notag \\
		&=\left[ \mathcal{P}_gd_i\left( 1-\mathcal{P}_g \right) \right] \hat{H}_{\text{at}}^{-1}\left( 1-\mathcal{P}_g \right) d_j\mathcal{P}_g \notag \\
		&=\mathcal{P}_gd_i\hat{H}_{\text{at}}^{-1}\left[ \left( 1-\mathcal{P}_g \right) d_j\mathcal{P}_g \right]  \notag \\
		&=\mathcal{P}_gd_i\hat{H}_{\text{at}}^{-1}d_j\mathcal{P}_g \tag{1}
	\end{align}
\end{tcolorbox}

\textbf{The Scalar Light Shift. }我们首先考虑 \(\alpha_\text{f} = 0\) 的特例;此时 \(\hat{H}_{\text{at}}\) 简化为 \(\Delta_e P_e\),并且
\begin{align}\label{eq:1.25}
	\hat{\mathcal{D}}_{ij} = \frac{1}{\Delta_e} \mathcal{P}_g d_i d_j \mathcal{P}_g
\end{align}
由于基态的 \(L = 0\),所以它具有空间反射对称性和旋转对称性,如果 \(i \neq j\),则 \(\hat{\mathcal{D}}_{ij} = 0\),而三个 \(\hat{\mathcal{D}}_{jj}\)(\(j = x, y, z\))都是相同的。因此,
\begin{align}\label{eq:1.26}
	\hat{\mathcal{D}}_{ij} = -4 u_s \delta_{ij}, \quad u_s = -\frac{e^2}{12 \Delta_e} \langle g | r^2 | g \rangle
\end{align}
这会产生一个有效哈密顿量
\begin{align}\label{eq:1.27}
	\hat{H}_{\text{eff}} = u_s \mathcal{E}^2
\end{align}
\begin{tcolorbox}[colback=red!3, colframe=red!60!black, title=注释, breakable]
	由式 \ref{eq:1.26} 有
	\begin{align}
		\hat{\mathcal{D}}_{ij}&=-4u_s\delta _{ij} \notag \\
		&=-4\left( -\frac{e^2}{12\Delta _e}\left< g \right|r^2\left| g \right> \right) \delta _{ij} \notag \\
		&=\frac{1}{3}\frac{1}{\Delta _e}\left< g \right|\left( er \right) ^2\left| g \right> \delta _{ij} \notag \\
		&=\frac{1}{3}\sum_j{\hat{\mathcal{D}}_{jj}^{2}}\delta _{ij}=\frac{1}{3}\hat{\mathcal{D}}^2\delta _{ij} \tag{1}
	\end{align}
	将式 \ref{eq:1.26} 带入式 \ref{eq:1.23} 有
	\begin{align}
		\hat{H}_{\text{eff}}&=-\frac{1}{4}\sum_{i,j}{\mathcal{E}_{i}^{*}\hat{\mathcal{D}}_{ij}}\mathcal{E}_j \notag \\
		&=-\frac{1}{4}\sum_{i,j}{\mathcal{E}_{i}^{*}\left( -4u_s\delta _{ij} \right)}\mathcal{E}_j \notag \\
		&=-\frac{1}{4}\sum_i{\left( -4u_s \right) \mathcal{E}_{i}^{2}} \notag \\
		&=u_s\sum_i{\mathcal{E}_{i}^{2}} \notag \\
		&=u_s\mathcal{E}^2 \tag{2}
	\end{align}
\end{tcolorbox}

这一项对基态中的不同自旋态具有相同的作用,不能翻转自旋,并且这一项也不依赖于光的偏振。这是因为偶极耦合仅作用在价电子的轨道自由度上,在没有 \(\mathbf{\hat{S}}\cdot \mathbf{\hat{L}}\) 耦合的情况下,轨道自由度与自旋自由度解耦。因此这一项被称为\CJKunderdot{\textit{标量光移}}(scalar light shift),\(u_s\) 被称为原子的\CJKunderdot{\textit{标量交流极化率}}(scalar ac polarizability)。下面我们将讨论标量光移的一些重要应用。

\textbf{Trapping Atoms with Laser. }标量势[式 \ref{eq:1.27}]只取决于光的强度。对于红失谐情况,\(u_s < 0\),原子将被束缚在光强为极大值的地方。因此,我们可以通过聚焦光束来捕获原子。这就是激光俘获原子的基本机制。在这种光学陷阱中,基态中的所有自旋态将感受到相同的势,因此可以充分利用原子的自旋自由度,正如我们将在第 \ref{sec:Spinor Condensate} 节中讨论的那样。

到目前为止,我们考虑的是吸收光子后伴随\CJKunderdot{\textit{受激辐射}}(stimulated emission)的二阶过程。实际上,还有另一种过程,即吸收光子后伴随\CJKunderdot{\textit{自发辐射}}(spontaneous emission)。自发辐射是使得激发态具有有限寿命的原因,可以通过在激发态能量中添加一个虚部 \(\Gamma\) 来描述(即 $ E_{\text{ex}}-i\varGamma  $),其中 \(\Gamma\) 是激发态的线宽。因此,这个过程可以通过在交流极化率中为激发态能量添加一个虚部来描述。添加了一个虚部的 \(u_s\) 为
\begin{align}\label{eq:1.28}
	u_s\propto \frac{1}{\Delta _e-i\Gamma}=\underset{optical\ trapping}{\underbrace{\frac{\Delta _e}{\Delta _{e}^{2}+\Gamma ^2}}}+\underset{laser\ cooling}{\underbrace{i\frac{\Gamma}{\Delta _{e}^{2}+\Gamma ^2}}}
\end{align}
对于失谐 \(\Delta_e \gg \Gamma\) 的情况,实部正比于 \(1/\Delta_e\),而虚部正比于 \(\Gamma/\Delta_e^2\)。由于实部和虚部依赖于 \(\Delta_e\) 的不同幂次,因此在\CJKunderdot{\textit{远失谐}}(far-detuned)区域中,虚部会远小于实部,而实部的幂次依赖将产生俘获效应,因此用激光俘获原子通常是在远失谐区域。

\textbf{Optical Lattice. }现在我们考虑两束在 \(\hat{x}\) 方向上相向传播的激光。这两束光具有相同的频率和偏振,例如两束光都沿 \(\hat{y}\) 方向线性偏振。假设这两束光的强度都为 \(E_0\);根据式 \ref{eq:1.22},我们有
\begin{align}\label{eq:1.29}
	\mathcal{E}_y = E^0 e^{ikx} + E^0 e^{-ikx} = 2E^0 \cos(kx)
\end{align}

\begin{tcolorbox}[colback=red!3, colframe=red!60!black, title=注释, breakable]
	由式 \ref{eq:1.22} 有
	\begin{align}
		\mathcal{E}_y&=E_{y,1}^{0}e^{i\phi _{y,1}-i\left( \omega _1-\omega \right) t}+E_{y,2}^{0}e^{i\phi _{y,2}-i\left( \omega _2-\omega \right) t} \notag \\
		&=E^0e^{i\phi _{y,1}-i\left( \omega -\omega \right) t}+E^0e^{i\phi _{y,2}-i\left( \omega -\omega \right) t} \notag \\
		&=E^0e^{i\phi _{y,1}}+E^0e^{i\phi _{y,2}} \notag \\
		&=E^0e^{ikx}+E^0e^{-ikx} \notag \\
		&=2E^0\cos \left( kx \right) \tag{1}
	\end{align}
\end{tcolorbox}

电场强度因两束激光的干涉而产生空间周期性分布。因此标量光移产生了一个周期性晶格势 \(V(x) \propto \cos^2(x)\),即如今广为人知的光晶格。

\textbf{Cooling Atoms with Laser. }如果 \(\Delta_e\) 变小,标量势的虚部也变得重要,这意味着伴随自发辐射的过程变得重要。受激辐射过程和自发辐射过程有一个重要的区别。对于前者,因为吸收和辐射的是同一个光子,动量转移被抵消,所以原子在经历整个二阶过程后没有接收到光子动量。但对于后者,由于在自发辐射过程中光子可以向任何方向发射,因此平均而言,自发辐射过程中的动量转移为零,于是原子在吸收光子时会受到光场中光子的冲击。实际上,原子感受到的力正比于 \(\Gamma/\Delta_e^2\) 和光子的动量。现在考虑两束具有相同频率的相向传播的激光,如果一个原子静止不动,来自左侧激光的力与来自右侧激光的力相互抵消,因为从左侧和右侧激光接收到的动量转移大小相等符号相反。然而,如果一个原子向右运动,由于多普勒效应,两束激光的失谐 \(\Delta_e\) 取决于原子的速度,并且它们是不同的。因此,两侧的力不能互相抵消,可以证明净力总是与原子的速度方向相反。因此,这种效应减慢了原子的运动。这就是\CJKunderdot{\textit{激光冷却}}(laser cooling)的基本机制,它是使原子冷却至量子简并的最重要步骤之一。

然而,由于各种限制,仅通过激光冷却来直接达到原子气体的量子简并非常困难。通过激光冷却直接达到量子简并是在实现原子气体的玻色-爱因斯坦凝聚之后20多年才首次实现的。在大多数实验中,需要在激光冷却之后进行\CJKunderdot{\textit{蒸发冷却}}(evaporative cooling)。蒸发冷却的基本思想非常简单。考虑具有有限深度 \(U_0\) 的势阱,动能大于 \(U_0\) 的原子将有很大机会从阱中逃离。现在让我们逐渐降低 \(U_0\),于是越来越多具有较大动能的原子逃离势阱,剩余原子的平均动能降低。因此,剩余原子气体的温度降低。然而,我们应注意到,蒸发冷却是通过损失原子来降低温度的。随着原子的流失,简并温度也会降低。因此,蒸发冷却的效率变得非常重要。也就是说,降温速度必须快于简并温度降低的速度;否则,系统永远无法达到量子简并。蒸发冷却的效率显然取决于如何随时间 \(t\) 降低势阱深度 \(U_0\),控制这个动态过程是通过蒸发冷却达到量子简并的关键挑战。

\textbf{Vector Light Shift. }现在让我们考虑具有 \( \mathbf{\hat{S}}\cdot \mathbf{\hat{L}} \) 耦合的情况。正如我们上面讨论的,精细结构分裂 \(\alpha_\text{f}\) 可以与失谐 \(\Delta_e\) 相当。在这里,为了简化计算,我们考虑 \(\alpha_\text{f} / \Delta_e \ll 1\) 的情况,且当 \(\alpha_\text{f} \sim \Delta_e\) 时主要结论也不会改变。当 \(\alpha_\text{f} / \Delta_e \ll 1\) 时,我们可以将 \(\hat{H}_{\text{at}}^{-1}\) 按 \(\alpha_\text{f} / \Delta_e\) 展开,得到
\begin{align}\label{eq:1.30}
	\hat{H}_\text{at}^{-1} = \left( \frac{1}{\Delta_e} - \frac{\alpha_\text{f}}{\Delta_e^2} \hat{\mathbf{S}} \cdot \hat{\mathbf{L}} \right) \mathcal{P}_e
\end{align}

\begin{tcolorbox}[colback=red!3, colframe=red!60!black, title=注释, breakable]
	由 $ \frac{1}{1+x}=1-x $ 有
	\begin{align}
		\hat{H}_{\text{at}}^{-1}&=\frac{1}{\Delta _{\text{e}}\mathcal{P}_e+\alpha _{\text{f}}\mathbf{\hat{S}}\cdot \mathbf{\hat{L}}}=\frac{1}{\Delta _{\text{e}}}\frac{1}{\mathcal{P}_e+\frac{\alpha _{\text{f}}}{\Delta _{\text{e}}}\mathbf{\hat{S}}\cdot \mathbf{\hat{L}}}=\frac{1}{\Delta _{\text{e}}}\left( 1-\frac{\alpha _{\text{f}}}{\Delta _{\text{e}}}\mathbf{\hat{S}}\cdot \mathbf{\hat{L}} \right) \mathcal{P}_e \notag \\
		&=\left( \frac{1}{\Delta _{\text{e}}}-\frac{\alpha _{\text{f}}}{\Delta _{\text{e}}^{2}}\mathbf{\hat{S}}\cdot \mathbf{\hat{L}} \right) \mathcal{P}_e \tag{1}
	\end{align}
\end{tcolorbox}

因此
\begin{align}\label{eq:1.31}
	\hat{\mathcal{D}}_{ij} = \mathcal{P}_g d_i \frac{1}{\Delta_e} d_j \mathcal{P}_g - \frac{\alpha_\text{f}}{\Delta_e^2} \mathcal{P}_g d_i (\hat{\mathbf{S}} \cdot \hat{\mathbf{L}}) d_j \mathcal{P}_g
\end{align}

\begin{tcolorbox}[colback=red!3, colframe=red!60!black, title=注释, breakable]
	将式 \ref{eq:1.30} 带入式 \ref{eq:1.24} 有
	\begin{align}
		\hat{\mathcal{D}}_{ij}&=\mathcal{P}_gd_i\hat{H}_{\text{at}}^{-1}d_j\mathcal{P}_g =\mathcal{D}_{ij}^{\left( \text{s} \right)}+\mathcal{D}_{ij}^{\left( \text{v} \right)} \notag \\
		&=\mathcal{P}_gd_i\left( \frac{1}{\Delta _{\text{e}}}-\frac{\alpha _{\text{f}}}{\Delta _{\text{e}}^{2}}\mathbf{\hat{S}}\cdot \mathbf{\hat{L}} \right) \mathcal{P}_ed_j\mathcal{P}_g \notag \\
		&=\mathcal{P}_gd_i\frac{1}{\Delta _{\text{e}}}\mathcal{P}_ed_j\mathcal{P}_g-\frac{\alpha _{\text{f}}}{\Delta _{\text{e}}^{2}}\mathcal{P}_gd_i\left( \mathbf{\hat{S}}\cdot \mathbf{\hat{L}} \right) \mathcal{P}_ed_j\mathcal{P}_g \notag \\
		&=\mathcal{P}_gd_i\frac{1}{\Delta _{\text{e}}}\left( \mathcal{P}_e+\mathcal{P}_g \right) d_j\mathcal{P}_g-\frac{\alpha _{\text{f}}}{\Delta _{\text{e}}^{2}}\mathcal{P}_gd_i\left( \mathbf{\hat{S}}\cdot \mathbf{\hat{L}} \right) \left( \mathcal{P}_e+\mathcal{P}_g \right) d_j\mathcal{P}_g \notag \\
		&=\mathcal{P}_gd_i\frac{1}{\Delta _{\text{e}}}d_j\mathcal{P}_g-\frac{\alpha _{\text{f}}}{\Delta _{\text{e}}^{2}}\mathcal{P}_gd_i\left( \mathbf{\hat{S}}\cdot \mathbf{\hat{L}} \right) d_j\mathcal{P}_g \tag{1}
	\end{align}
	得到式 \ref{eq:1.31}。
\end{tcolorbox}

上面第一项给出了标量部分[式 \ref{eq:1.25}]。而对于第二项,有
\begin{align}
	\mathcal{P}_g d_i (\hat{\mathbf{S}} \cdot \hat{\mathbf{L}}) d_j \mathcal{P}_g &= \mathcal{P}_g (\hat{\mathbf{S}} \cdot \hat{\mathbf{L}}) d_i d_j \mathcal{P}_g - \mathcal{P}_g [(\hat{\mathbf{S}} \cdot \hat{\mathbf{L}}), d_i] d_j \mathcal{P}_g \label{eq:1.32} \\
	&= -i\hbar \epsilon_{lim} S_l \mathcal{P}_g d_m d_j \mathcal{P}_g \label{eq:1.33}
\end{align}

\begin{tcolorbox}[colback=red!3, colframe=red!60!black, title=注释, breakable]
	将对易关系
	\begin{align}
		d_i\left( \mathbf{\hat{S}}\cdot \mathbf{\hat{L}} \right) =\left( \mathbf{\hat{S}}\cdot \mathbf{\hat{L}} \right) d_i-\left[ \mathbf{\hat{S}}\cdot \mathbf{\hat{L},}d_i \right] \tag{1}
	\end{align}
	带入式 \ref{eq:1.31} 的第二项
	\begin{align}
		\mathcal{P}_g\left[ d_i\left( \mathbf{\hat{S}}\cdot \mathbf{\hat{L}} \right) \right] d_j\mathcal{P}_g&=\mathcal{P}_g\left\{ \left( \mathbf{\hat{S}}\cdot \mathbf{\hat{L}} \right) d_i-\left[ \left( \mathbf{\hat{S}}\cdot \mathbf{\hat{L}} \right) ,d_i \right] \right\} d_j\mathcal{P}_g \notag \\
		&=\mathcal{P}_g\left( \mathbf{\hat{S}}\cdot \mathbf{\hat{L}} \right) d_id_j\mathcal{P}_g-\mathcal{P}_g\left[ \left( \mathbf{\hat{S}}\cdot \mathbf{\hat{L}} \right) ,d_i \right] d_j\mathcal{P}_g \tag{2}
	\end{align}
	得到式 \ref{eq:1.32}。
\end{tcolorbox}

其中由于 $ \mathbf{\hat{L}}\mathcal{P}_g=0 $ 所以第一项为零,对于第二项,我们利用了坐标和角动量的对易关系 \([\hat{L}_l, \hat{d}_i] = i\hbar \epsilon_{lim} \hat{d}_m\)。因此我们得到

\begin{tcolorbox}[colback=red!3, colframe=red!60!black, title=注释, breakable]
	将对易关系
	\begin{align}
		\left[ \mathbf{\hat{S}}\cdot \mathbf{\hat{L},}d_i \right] =\left[ \hat{S}_l\hat{L}_l,d_i \right] =\hat{S}_l\left[ \hat{L}_l,d_i \right] =\hat{S}_l\left( i\hbar\epsilon _{lim}d_m \right) \tag{1}
	\end{align}
	带入式 \ref{eq:1.32} 的第二项
	\begin{align}
		\mathcal{P}_g\left[ d_i\left( \mathbf{\hat{S}}\cdot \mathbf{\hat{L}} \right) \right] d_j\mathcal{P}_g &=\mathcal{P}_g\left( \mathbf{\hat{S}}\cdot \mathbf{\hat{L}} \right) d_id_j\mathcal{P}_g-\mathcal{P}_g\left[ \left( \mathbf{\hat{S}}\cdot \mathbf{\hat{L}} \right) ,d_i \right] d_j\mathcal{P}_g \notag \\
		&=-\mathcal{P}_g\left[ \left( \mathbf{\hat{S}}\cdot \mathbf{\hat{L}} \right) ,d_i \right] d_j\mathcal{P}_g \notag \\
		&=-\mathcal{P}_g\hat{S}_l\left( i\hbar\epsilon _{lim}d_m \right) d_j\mathcal{P}_g \notag \\
		&=-i\hbar\epsilon _{lim}\hat{S}_l\mathcal{P}_gd_md_j\mathcal{P}_g \tag{2}
	\end{align}
	得到式 \ref{eq:1.33}。
\end{tcolorbox}

\begin{align}\label{eq:1.34}
	\hat{\mathcal{D}}_{ij} = -4u_s \delta_{ij} + i \frac{\hbar \alpha_\text{f}}{\Delta_e} \epsilon_{ijl} S_l (-4u_s) \simeq -4u_s \left( \delta_{ij} + i \frac{\hbar \alpha_\text{f}}{\Delta_e} \epsilon_{ijl} S_l \right)
\end{align}

\begin{tcolorbox}[colback=red!3, colframe=red!60!black, title=注释, breakable]
	将式 \ref{eq:1.26} 和式 \ref{eq:1.33} 带入式 \ref{eq:1.31},有
	\begin{align}
		\hat{\mathcal{D}}_{ij}&=\mathcal{P}_gd_i\frac{1}{\Delta _e}d_j\mathcal{P}_g-\frac{\alpha _{\text{f}}}{\Delta _{e}^{2}}\mathcal{P}_gd_i\left( \mathbf{\hat{S}}\cdot \mathbf{\hat{L}} \right) d_j\mathcal{P}_g \notag \\
		&=-4u_s\delta _{ij}-\frac{\alpha _{\text{f}}}{\Delta _{e}^{2}}\left( -i\hbar\epsilon _{lim}\hat{S}_l\mathcal{P}_gd_md_j\mathcal{P}_g \right)  \notag \\
		&=-4u_s\delta _{ij}+\frac{\alpha _{\text{f}}}{\Delta _{e}^{2}}i\hbar\epsilon _{lim}\hat{S}_l\mathcal{P}_gd_md_j\mathcal{P}_g \notag \\
		&=-4u_s\delta _{ij}+\frac{\alpha _{\text{f}}}{\Delta _{\text{e}}}i\hbar\epsilon _{lim}\hat{S}_l\left( \frac{1}{\Delta _{\text{e}}}\mathcal{P}_gd_md_j\mathcal{P}_g \right)  \notag \\
		&=-4u_s\delta _{ij}+i\frac{\hbar \alpha _{\text{f}}}{\Delta _{\text{e}}}\epsilon _{lim}\hat{S}_l\left( -4u_s\delta _{mj} \right)  \notag \\
		&=-4u_s\delta _{ij}+i\frac{\hbar\alpha _{\text{f}}}{\Delta _{\text{e}}}\epsilon _{lij}\hat{S}_l\left( -4u_s \right)  \notag \\
		&=-4u_s\delta _{ij}+i\frac{\hbar\alpha _{\text{f}}}{\Delta _{\text{e}}}\epsilon _{ijl}\hat{S}_l\left( -4u_s \right)  \notag \\
		&\approx -4u_s\left( \delta _{ij}+i\frac{\hbar\alpha _{\text{f}}}{\Delta _{\text{e}}}\epsilon _{ijl}\hat{S}_l \right) \tag{1}
	\end{align}
	得到式 \ref{eq:1.34}。
\end{tcolorbox}
其中第二项将给出矢量光频移。在这种情况下,有效哈密顿量为
\begin{align}\label{eq:1.35}
	\hat{H}_{\text{eff}} = -\frac{1}{4} \sum_{i,j=x,y,z} \mathcal{E}_i^* \hat{D}_{ij} \mathcal{E}_j = u_s |\mathcal{E}|^2 + i u_v (\mathcal{E}^* \times \mathcal{E}) \cdot \hat{\mathbf{S}}
\end{align}
其中 \(u_v = \hbar \alpha_\text{f} u_s / \Delta_e\) 是\CJKunderdot{\textit{矢量极化率}}(vector polarizability)。矢量光频移的物理意义是产生一个作用于碱金属原子电子自旋的塞曼场 \(\mathbf{B} = iu_v \mathbf{\mathcal{E}}^* \times \mathbf{\mathcal{E}}\)。

\begin{tcolorbox}[colback=red!3, colframe=red!60!black, title=注释, breakable]
	将式 \ref{eq:1.34} 带入式 \ref{eq:1.23},有
	\begin{align}
		\hat{H}_{\text{eff}}&=-\frac{1}{4}\sum_{i,j}{\mathcal{E}_{i}^{*}\left[ -4u_s\left( \delta _{ij}+i\frac{\hbar\alpha _{\text{f}}}{\Delta _{\text{e}}}\epsilon _{ijl}\hat{S}_l \right) \right] \mathcal{E}_j} \notag \\
		&=\sum_{i,j}{\mathcal{E}_{i}^{*}\left[ u_s\left( \delta _{ij}+i\frac{\hbar\alpha _{\text{f}}}{\Delta _{\text{e}}}\epsilon _{ijl}\hat{S}_l \right) \right] \mathcal{E}_j} \notag \\
		&=\sum_{i,j}{\mathcal{E}_{i}^{*}u_s\delta _{ij}\mathcal{E}_j}+\sum_{i,j,l}{\mathcal{E}_{i}^{*}u_s\left( i\frac{\hbar\alpha _{\text{f}}}{\Delta _{\text{e}}}\epsilon _{ijl}\hat{S}_l \right) \mathcal{E}_j} \notag \\
		&=\sum_i{u_s\mathcal{E}_{i}^{*}\mathcal{E}_i}+u_si\frac{\hbar\alpha _{\text{f}}}{\Delta _{\text{e}}}\left( \mathcal{E}^*\times \mathcal{E} \right) \cdot \mathbf{\hat{S}} \notag \\
		&=u_s\left| \mathcal{E} \right|^2+i\left( u_s\frac{\hbar\alpha _{\text{f}}}{\Delta _{\text{e}}} \right) \left( \mathcal{E}^*\times \mathcal{E} \right) \cdot \mathbf{\hat{S}} \notag \\
		&=u_s\left| \mathcal{E} \right|^2+\underset{Zeeman\ Field}{\underbrace{iu_v\left( \mathcal{E}^*\times \mathcal{E} \right) }}\cdot \mathbf{\hat{S}} \tag{1}
	\end{align}
	得到式 \ref{eq:1.35}。
\end{tcolorbox}

同样,因为偶极耦合作用于轨道自由度,而矢量光频移依赖于耦合轨道自由度与电子自旋自由度的 \(\mathbf{\hat{S}} \cdot \mathbf{\hat{L}}\)。在这种情况下,由于 \(\mathbf{\hat{S}} \cdot \mathbf{\hat{L}}\) 只存在于激发态中,故矢量光频移比标量小 \(\hbar \alpha_\text{f} / \Delta_e\) 的量级。因此矢量光频移也按 \(1/\Delta_e^2\) 随 \(\Delta_e\) 变化。因此,矢量光频移与自发辐射过程具有相同的量级。在讨论标量光频移时,我们注意到,由于标量光频移和自发辐射对 \(\Delta_e\) 具有不同的幂次,可以通过增加失谐来抑制自发辐射对于标量势的影响。然而,对于矢量光频移这种方法不再有效。因为矢量光频移和自发辐射均以相同的方式随 \(\Delta_e\) 变化,所以无法在不降低矢量光频移强度的情况下抑制自发辐射过程。

\textbf{Light-Induced Zeeman Energy. }下面我们将讨论几个作为矢量光频移应用的例子。如果以线性偏振光照射原子,那么 \(\mathbf{\mathcal{E}}^* \times \mathbf{\mathcal{E}} = 0\),矢量光频移消失。如果光是圆偏振的,例如 \(\mathbf{\mathcal{E}} = \frac{E_0}{\sqrt{2}} (\mathbf{e}_\mathbf{x} + i\mathbf{e}_\mathbf{y})\),那么
\begin{align}\label{eq:1.36}
	\mathcal{E}^* \times \mathcal{E} = i E_0^2 \mathbf{e}_\mathbf{z}
\end{align}
\begin{tcolorbox}[colback=red!3, colframe=red!60!black, title=注释, breakable]
	\begin{align}
		\mathcal{E}^*\times \mathcal{E}&=\left[ \frac{E_0}{\sqrt{2}}\left( \mathbf{e}_{\mathbf{x}}+i\mathbf{e}_{\mathbf{y}} \right) \right] ^*\times \left[ \frac{E_0}{\sqrt{2}}\left( \mathbf{e}_{\mathbf{x}}+i\mathbf{e}_{\mathbf{y}} \right) \right]  \notag \\
		&=\frac{E_{0}^{2}}{2}\left( \mathbf{e}_{\mathbf{x}}-i\mathbf{e}_{\mathbf{y}} \right) \times \left( \mathbf{e}_{\mathbf{x}}+i\mathbf{e}_{\mathbf{y}} \right)  \notag \\
		&=\frac{E_{0}^{2}}{2}\left[ \mathbf{e}_{\mathbf{x}}\times \left( i\mathbf{e}_{\mathbf{y}} \right) +\left( -i\mathbf{e}_{\mathbf{y}} \right) \times \mathbf{e}_{\mathbf{x}} \right]  \notag \\
		&=\frac{E_{0}^{2}}{2}\left( i\mathbf{e}_{\mathbf{z}}+i\mathbf{e}_{\mathbf{z}} \right)  \notag \\
		&=iE_{0}^{2}\mathbf{e}_{\mathbf{z}}  \tag{1}
	\end{align}
	得到式 \ref{eq:1.36}。
\end{tcolorbox}
这会产生一个与 \(u_v E_0^2 \hat{S}_z\) 成正比的\CJKunderdot{\textit{光诱导塞曼能}}(light-induced Zeeman energy )。线性偏振光和圆偏振光之间的这种差异在于它们的对称性的差异。线性偏振光的存在不破坏时间反转对称性,但圆偏振光的存在则会破坏该对称性。由于塞曼场的存在会打破时间反转对称性,从对称性角度来看,它与圆偏振光的存在是相容的,但与线性偏振光的存在不相容。

由于 \(u_v\) 取决于精细结构分裂和失谐,如果两个不同的原子被同一圆偏振光照射,它们将感受到不同的光诱导塞曼场。一般来说,由于较重原子的自旋轨道耦合更强,对于给定的失谐,较重原子的矢量光频移也更大。一个可以展示这种效应的实验是 \(^{23}\text{Na}\) 和 \(^{87}\text{Rb}\) 原子之间的\CJKunderdot{\textit{自旋交换散射}}(spin exchanging scattering)。自旋交换散射将在第 \ref{sec:Spin-Dependent Interaction} 节中详细讨论。这里我们先简要提及。考虑 \(^{23}\text{Na}\) 和 \(^{87}\text{Rb}\) 原子的 \(F = 1\) 自旋态,每个自旋态都有三个用 \(|F_z\rangle\) 标记的磁子态。例如,一个自旋交换散射过程可以发生如下:
\begin{align}\label{eq:1.37}
	|0\rangle_{\text{Na}} | -1\rangle_{\text{Rb}} \iff | -1\rangle_{\text{Na}} | 0\rangle_{\text{Rb}}
\end{align}
当入射态和出射态之间的塞曼能量差为零时,这种自旋交换散射在某一磁场下达到共振,也就是说
\begin{align}\label{eq:1.38}
	\Delta E = \left( E_{|0\rangle_{\text{Na}}} + E_{|-1\rangle_{\text{Rb}}} \right) - \left( E_{|-1\rangle_{\text{Na}}} + E_{|0\rangle_{\text{Rb}}} \right) = 0 
\end{align}
此时自旋交换振荡的周期达到最大。塞曼能量包括静磁场和光诱导塞曼场的贡献。在通常的实验条件下,光诱导塞曼能量对于 \(^{87}\text{Rb}\) 相当于施加了约 mG 的磁场,但对于 \(^{23}\text{Na}\) 的光诱导塞曼能量仅相当于施加了约 $ \mu  $G 的磁场,这可以忽略不计。这种差异是由于它们的精细结构分裂差异。因此,在不同的光强下达到自旋交换共振的磁场强度也不同。这已经在实验中被观察到,实验结果显示在图 \ref{fig:1.6} 中。

\begin{figure}[h!]
	\centering
	\includegraphics[width=8cm]{images/1-6.png}  % 指定相对路径
	\caption{光诱导塞曼能量的物理效应。(a) 不同自旋态之间的能量差 \(\Delta E\) 由式 \ref{eq:1.38} 定义。(b) 实验观测到的自旋交换振荡周期随磁场强度的变化关系(针对不同的光诱导磁场)。图中的 \(\text{B}_{\text{ac}}\) 表示在不同强度的圆偏振光下 \(^{87}\text{Rb}\) 的光诱导塞曼场。}
	\label{fig:1.6}
\end{figure}

\begin{figure}[h!]
	\centering
	\includegraphics[width=12cm]{images/1-7.png}  % 指定相对路径
	\caption{人工自旋轨道耦合的示意图。(a) 拉曼耦合的激光构型。两个具有不同偏振方向和不同频率的相向传播的拉曼光束沿 \(\hat{x}\) 方向入射,\( \mathbf{B }\) 是沿 \(\hat{z}\) 方向施加的真实磁场。(b) 拉曼耦合将两个间距为 \(h\) 的塞曼能级耦合了起来。}
	\label{fig:1.7}
\end{figure}

\textbf{Synthetic Spin-Orbit Coupling. }现在我们考虑处于沿 \(\hat{z}\) 方向的静磁场中的原子,该磁场产生的塞曼能量假设为 \(hF_z\) 以简化问题。此外,我们考虑沿 \(\hat{x}\) 方向的两束相向传播的\CJKunderdot{\textit{拉曼光束}}(Raman beams),其中一个沿 \(\hat{y}\) 方向偏振,频率为 \(\omega_1\),另一个沿 \(\hat{z}\) 方向偏振,频率为 \(\omega_2\),如图 \ref{fig:1.7} 所示。电场由下式给出
\begin{align}\label{eq:1.39}
	\mathbf{E} = E_1 e^{ik_0 x + i \omega_1 t} \mathbf{e}_\mathbf{y} + E_2 e^{-ik_0 x + i \omega_2 t} \mathbf{e}_\mathbf{z}
\end{align}
这是两束不同频率激光的情况。借助式 \ref{eq:1.22},可以得到
\begin{align}\label{eq:1.40}
	\mathcal{E}^* \times \mathcal{E} = E_1 E_2 \left( e^{-i 2 k_0 x - i \delta \omega t} - e^{i 2 k_0 x + i \delta \omega t} \right) \mathbf{e}_\mathbf{x}
\end{align}

\begin{tcolorbox}[colback=red!3, colframe=red!60!black, title=注释, breakable]
	由式 \ref{eq:1.22} 和式 \ref{eq:1.39} 得
	\begin{align}
		\mathcal{E}_x=0  \tag{1}
	\end{align}
	\begin{align}
		\mathcal{E}_y&=E_{y,1}^{0}e^{i\phi _{y,1}-i\left( \omega _1-\omega \right) t}+E_{y,2}^{0}e^{i\phi _{y,2}-i\left( \omega _2-\omega \right) t} \notag \\
		&=E_1e^{ik_0x-i\left( \omega _1-\omega \right) t}+0 \notag \\
		&=E_1e^{ik_0x-i\left( \omega _1-\omega \right) t}  \tag{2}
	\end{align}
	\begin{align}
		\mathcal{E}_z&=E_{z,1}^{0}e^{i\phi _{z,1}-i\left( \omega _1-\omega \right) t}+E_{z,2}^{0}e^{i\phi _{z,2}-i\left( \omega _2-\omega \right) t} \notag \\
		&=0+E_2e^{-ik_0x-i\left( \omega _2-\omega \right) t} \notag \\
		&=E_2e^{-ik_0x-i\left( \omega _2-\omega \right) t}  \tag{3}
	\end{align}
	将上面三式带入式 \ref{eq:1.40} 左边
	\begin{align}
		&\mathcal{E}^*\times \mathcal{E}=\left( \mathcal{E}_{y}^{*}\mathcal{E}_z-\mathcal{E}_{z}^{*}\mathcal{E}_y \right) \mathbf{e}_{\mathbf{x}} \notag \\
		&=\left[ \left( E_1e^{ik_0x-i\left( \omega _1-\omega \right) t} \right) ^*\left( E_2e^{-ik_0x-i\left( \omega _2-\omega \right) t} \right) -\left( E_2e^{-ik_0x-i\left( \omega _2-\omega \right) t} \right) ^*\left( E_1e^{ik_0x-i\left( \omega _1-\omega \right) t} \right) \right] \mathbf{e}_{\mathbf{x}} \notag \\
		&=\left[ E_1E_2e^{-2ik_0x-i\left( \omega _2-\omega _1 \right) t}-E_1E_2e^{2ik_0x+i\left( \omega _2-\omega _1 \right) t} \right] \mathbf{e}_{\mathbf{x}} \notag \\
		&=E_1E_2\left( e^{-2ik_0x-i\delta \omega t}-e^{2ik_0x+i\delta \omega t} \right) \mathbf{e}_{\mathbf{x}}  \tag{4}
	\end{align}
	得到式 \ref{eq:1.40} 右边。
\end{tcolorbox}

其中 \(\delta\omega = \omega_2 - \omega_1\)。然后,处于该光场中的原子会感受到一个光诱导塞曼场:
\begin{align}\label{eq:1.41}
	\hat{H}_{\text{eff}} = i u_v E_1 E_2 \left( e^{-i 2 k_0 x - i \delta \omega t} - e^{i 2 k_0 x + i \delta \omega t} \right) \hat{S}_x
\end{align}

\begin{tcolorbox}[colback=red!3, colframe=red!60!black, title=注释, breakable]
	将式 \ref{eq:1.40} 带入式 \ref{eq:1.35} 得
	\begin{align}
		\hat{H}_{\text{eff}}&=iu_v\left( \mathcal{E}^*\times \mathcal{E} \right) \cdot \mathbf{\hat{S}} \notag \\
		&=iu_v\left[ E_1E_2\left( e^{-2ik_0x-i\delta \omega t}-e^{2ik_0x+i\delta \omega t} \right) \mathbf{e}_{\mathbf{x}} \right] \cdot \mathbf{\hat{S}} \notag \\
		&=iu_vE_1E_2\left( e^{-2ik_0x-i\delta \omega t}-e^{2ik_0x+i\delta \omega t} \right) \hat{S}_x  \tag{1}
	\end{align}
	得到式 \ref{eq:1.41}。
\end{tcolorbox}

在弱场区域,利用 Wigner–Eckart 定理,可以将哈密顿量[式 \ref{eq:1.41}]投影到超精细本征态基矢中,此时 \(\hat{S}_{x,y,z}\) 分别变为 \(\hat{F}_{x,y,z}\)。因此,自旋部分的总哈密顿量包含真实的静磁场和光诱导塞曼场
\begin{align}\label{eq:1.42}
	\hat{H}_\text{s} = h \hat{F}_z + i \Omega \left( e^{-i 2 k_0 x - i \delta \omega t} - e^{i 2 k_0 x + i \delta \omega t} \right) \hat{F}_x
\end{align}
\begin{tcolorbox}[colback=red!3, colframe=red!60!black, title=注释, breakable]
	根据 Wigner–Eckart 定理,从 \(\hat{S}_{x,y,z}\) 投影到 \(\hat{F}_{x,y,z}\) 会多出一个常系数,该系数为
	\begin{align}
		\frac{F\left( F+1 \right) -I\left( I+1 \right) +J\left( J+1 \right)}{2F\left( F+1 \right)}  \tag{1}
	\end{align}
	对于 \(^{87}\text{Rb}\),\(I = 3/2\),\(J = 1/2\),取 \(F = 1\),这个常数是 \(-1/4\)。
\end{tcolorbox}
其中 \(\Omega\) 表示正比于 \(E_1E_2\) 的耦合常数。我们考虑 \(\delta\omega\) 和 \(h\) \CJKunderdot{\textit{近共振}}(near-resonant)的情况。根据上述讨论,我们作用一个幺正变换 \(\hat{\mathcal{U}} = e^{-i\delta\omega t \hat{F}_z / \hbar}\) 并用旋转波近似扔掉具有 \(2\delta\omega\) 频率的项,哈密顿量可以约化为
\begin{align}\label{eq:1.43}
	\hat{H}_\text{s} = (h - \delta \omega) \hat{F}_z + \Omega (\sin(2k_0 x) \hat{F}_x - \cos(2k_0 x) \hat{F}_y)
\end{align}

\begin{tcolorbox}[colback=red!3, colframe=red!60!black, title=注释, breakable]
	\centering
    \includegraphics[width=0.3\textwidth]{images/chap1-1.png}
	\begin{align}
		&h-\omega _2+\omega _1\approx 0\Longrightarrow near\ resonant \tag{1} \\
		&h-\omega _1+\omega _2\approx 2\delta \omega \Longrightarrow off\ resonance \tag{2}
	\end{align}
\end{tcolorbox}
这个哈密顿量表示一个在空间中变化方向的塞曼场,其在光波长的尺度上改变自旋方向。

上述讨论主要涉及原子内电子的运动。现在,我们将原子视为在这样一个空间依赖的塞曼场中运动的点粒子,其哈密顿量表示为
\begin{align}\label{eq:1.44}
	\hat{H} = \frac{\hbar^2 \hat{\mathbf{k}}^2}{2m} + \hat{H}_\text{s}
\end{align}
其中 \(\mathbf{k}\) 现在代表原子的空间运动。通过对哈密顿量作用上一个空间依赖的自旋旋转 \(\hat{\mathcal{U}} = e^{-i 2k_0 x \hat{F}_z}\),可以得到
\begin{align}\label{eq:1.45}
	\hat{H} = \frac{\hbar^2}{2m} \left( \hat{k}_x - 2k_0 \hat{F}_z / \hbar \right)^2 + \frac{\hbar^2 \hat{\mathbf{k}}_\perp^2}{2m} + (h - \delta \omega) \hat{F}_z - \Omega \hat{F}_y
\end{align}

下面我们将分析模型 \ref{eq:1.45} 的几种不同情况。

当 \(h - \delta\omega\) 远大于 \(\Omega\) 时,我们可以通过绝热近似来只保留最低的自旋分支。不失一般性,我们考虑 \(h - \delta\omega > 0\) 的情况。在色散关系的最小值附近,色散可以很好地近似为 \(\frac{1}{2m^*}(k_x - k_{\text{min}})^2\)。对于小的 \(\Omega\),\(k_{\text{min}} \approx 2k_0 + o(\Omega)\)。这可以被认为实现了一个常数 \(U(1)\) 规范场 \( \hat{H} = \frac{1}{2m^*}(k_x - A_x)^2\),其中 \(A_x\) 是一个常数。一个常数的 \(U(1)\) 规范场没有物理效应,因为它既不产生电场也不产生磁场,并且它可以通过规范变换 \(e^{ik_{\text{min}}x}\) 来消去。

在上述讨论的区域,如果 \(\Omega\) 依赖于 \(y\) 坐标,这意味着 \(k_{\text{min}}\),或者更准确地说,\(A_x\),依赖于 \(y\)。这对应了一个人工磁场 \(B_{\text{syn}} = -\partial A_x/\partial y\)。这种涌现的人工磁场亦可理解为光诱导磁场使得自旋的取向被极化,如第 \ref{sec:Magnetic Structure} 节所讨论的那样。如果 \(\Omega\) 依赖于时间,则 \(A_x\) 也依赖于时间。这就实现了一个人工电场 \(E_{\text{syn}} = -\partial A_x/\partial t\)。然而,这些规范场的空间和时间依赖性完全由外部经典场(此处即指光强的分布)所决定。这些规范场本身不存在量子动力学。我们将在 Box 7.3 中总结模拟各种类型的 \(U(1)\) 规范场的时间线,并讨论如何实现具有自身动力学的 \(U(1)\) 规范场。

在 \(h - \delta\omega\) 与 \(\Omega\) 可比的区域,我们需要保留所有自旋分量。在这里,我们依然认为 \(\Omega\) 是一个不依赖于空间和时间坐标的常数,但我们需要将矢量规范势 \(A_x = 2k_0\hat{F}_z/\hbar\) 和标量势 \(-\Omega \hat{F}_y\) 作为矩阵保留。这两项彼此不对易,从而实现了一个非阿贝尔规范场。与阿贝尔情况不同,即使非阿贝尔规范场是常数,它们也具有物理效应。例如,如果我们想要消去矢量势,则需要作用上规范变换 \(e^{2ik_0 x \hat{F}_z}\),而这个规范变换与标量势不对易。

在这种情况下,哈密顿量[式 \ref{eq:1.45}]也可以写为
\begin{align}\label{eq:1.46}
	\hat{H} = \frac{\hbar^2 \hat{\mathbf{k}}^2}{2m} + \mathbf{h}_{\mathbf{k}} \cdot \hat{\mathbf{F}}
\end{align}
其中
\begin{align}\label{eq:1.47}
	\mathbf{h}_{\mathbf{k}} = \left( 0, -\Omega, \left( h - \delta \omega - \frac{\hbar^2}{m} k_0 k_x \right) \right)
\end{align}

\begin{tcolorbox}[colback=red!3, colframe=red!60!black, title=注释, breakable]
	\centering
    \includegraphics[width=0.8\textwidth]{images/chap1-2.png}
\end{tcolorbox}

这个形式的哈密顿量可以被看作一个动量依赖的塞曼场。对于其本征态,原子的自旋方向被其动量锁定,从而产生自旋-轨道耦合效应。在此需要注意,不应将这种人工的自旋-轨道耦合与原子结构中讨论的真实自旋-轨道耦合混淆。对于人工的自旋-轨道耦合,“轨道”指的是原子的空间运动,而“自旋”指的是原子的总自旋。在自然界中,中性原子本身并不具有自旋-轨道耦合效应,这里是由原子-光相互作用产生的一种效应。而真实的自旋-轨道耦合指的是自然界中由精细结构耦合产生的效应,其中“轨道”指的是电子在原子内部围绕原子核的运动,“自旋”指的是电子的自旋。然而,从上述讨论可以看出,这种人工的自旋-轨道耦合依赖于真实的自旋-轨道耦合。

我们在上述讨论中并未包含二次塞曼能量。考虑像 \(^{87}\text{Rb}\) 这样的原子,其 \(F = 1\),在适当的磁场下,还存在显著的二次塞曼能量,表示为 \(qF_z^2\),如在第 \ref{sec:Magnetic Structure} 节中讨论的那样。特别是,当 \(h - \delta\omega \approx q \gg \Omega\) 时,有
\begin{align}\label{eq:1.48-49}
	\delta = h - \delta \omega - q \ll \Omega \\
	h - \delta \omega + q \approx 2q \gg \Omega
\end{align}
在这种情况下,\(|F = 1, F_z = 0\rangle\) 和 \(|F = 1, F_z = -1\rangle\) 近乎简并,而它们与 \(|F = 1, F_z = 1\rangle\) 之间的能量间隔远大于它们之间的耦合。因此,我们引入一个赝自旋-1/2 来表示 \(|F = 1, F_z = 0\rangle\) 和 \(|F = 1, F_z = -1\rangle\),用泡利矩阵 \(\mathbf{\sigma }\) 来描述,并忽略 \(|F = 1, F_z = 1\rangle\)。在由泡利矩阵描述的自旋-1/2 子空间中,哈密顿量可以写为
\begin{align}\label{eq:1.50}
	\hat{H} = \frac{\hbar^2}{2m} \left( \hat{k}_x - k_0 \sigma_z \right)^2 + \frac{\hbar^2 \hat{\mathbf{k}}_\perp^2}{2m} + \frac{\delta}{2} \sigma_z - \Omega \sigma_y
\end{align}
其中,\(\delta\) 表示 \(h - \delta\omega - q\)。经过一次自旋翻转后,哈密顿量也可以写成文献中最常用的形式:
\begin{align}\label{eq:1.51}
	\hat{H} = \frac{\hbar^2}{2m} \left( \hat{k}_x - k_0 \sigma_z \right)^2 + \frac{\hbar^2 \hat{\mathbf{k}}_\perp^2}{2m} + \frac{\delta}{2} \sigma_z + \Omega \sigma_x
\end{align}
这个哈密顿量在玻色凝聚态和简并费米气体中都得到了实现。自旋-轨道耦合在超冷玻色气体和费米气体中的效应在冷原子物理中得到了广泛研究。我们将在第 \ref{sec:Spin-Orbit-Coupled Condensate} 节讨论这种人工自旋-轨道耦合如何影响玻色凝聚态的性质。



\subsection{Stimulated Raman Adiabatic Passage}\label{sec:Stimulated Raman Adiabatic Passage}
在第 \ref{sec:Light Shift} 节中,我们已经讨论了基态原子可以通过有中间激发态参与的双光子过程而感受到一个有效势。在本节中,我们将讨论原子从一个低能态转移到另一个低能态的动力学过程,这也是通过有中间态参与的双光子过程来实现的。这被称为\CJKunderdot{\textit{受激拉曼绝热传态}}(Stimulated Raman Adiabatic Passage,STIRAP) 。

\begin{figure}[h!]
	\centering
	\includegraphics[width=10cm]{images/1-8.png}  % 指定相对路径
	\caption{STIRAP 方案示意图。(a) 两束激光在三个不同阶段如何耦合三个不同的态。虚线表示斯托克斯光,实线表示泵浦光。(b) \(\Omega_\text{s}\)(虚线)和 \(\Omega_\text{p}\)(实线)随时间变化的曲线。(c) \(|1\rangle\) 态的布居(实线)和 \(|2\rangle\) 态的布居(虚线)随时间变化的曲线。}
	\label{fig:1.8}
\end{figure}

这里我们介绍最简单情况的 STIRAP。如图 \ref{fig:1.8} 所示,两个低能态分别用 \(|1\rangle\) 和 \(|2\rangle\) 表示,能量分别为 \(E_1\) 和 \(E_2\),激发态用 \(|e\rangle\) 表示,能量为 \(E_{\text{ex}}\)。一个称为“泵浦光”的激光将 \(|1\rangle\) 与 \(|e\rangle\) 耦合,能量为 \(\omega_\text{p}\) 且耦合强度为时间依赖的 \(\Omega_\text{p}(t)\),另一个称为“斯托克斯光”的激光将 \(|2\rangle\) 与 \(|e\rangle\) 耦合,能量为 \(\omega_\text{s}\) 且耦合强度为 \(\Omega_\text{s}(t)\)。需要注意的是,\(\Omega_\text{s}(t)\) 和 \(\Omega_\text{p}(t)\) 的时间依赖性应与其他时间尺度相比足够缓慢。我们引入两个失谐 \(\Delta_\text{p} = E_{\text{ex}} - E_1 - \hbar\omega_\text{p}\) 和 \(\Delta_\text{s} = E_{\text{ex}} - E_2 - \hbar\omega_\text{s}\),STIRAP 要求 \(\Delta_\text{p} = \Delta_\text{s}\),在下文中将它们表示为 \(\Delta\)。因此,在旋转波近似下,时间依赖的哈密顿量为:
\begin{align}\label{eq:1.52}
	\hat{H}(t) = \hbar |\Psi\rangle \mathcal{H}(t) \langle \Psi|^T
\end{align}
其中,\(|\Psi\rangle\) 表示 \((|1\rangle, |2\rangle, |e\rangle)\),而 \(\langle \Psi|^T\) 表示 \((\langle 1|, \langle 2|, \langle e|)^T\),且 \(\mathcal{H}(t)\) 是一个 \(3 \times 3\) 的矩阵,其形式如下:
\begin{align}\label{eq:1.53}
	\mathcal{H}(t) = \begin{pmatrix}
	0 & 0 & \Omega_\text{p}(t) \\
	0 & 0 & \Omega_\text{s}(t) \\
	\Omega_\text{p}(t) & \Omega_\text{s}(t) & \Delta
	\end{pmatrix}
\end{align}

这个哈密顿量实际上可以重写为
\begin{align}\label{eq:1.54}
	\hat{H}(t) = A(t) \left[ \left( \frac{\Omega_\text{p}(t)}{A(t)} |1\rangle + \frac{\Omega_\text{s}(t)}{A(t)} |2\rangle \right) \langle e| + \text{h.c.} \right] + \Delta |e\rangle \langle e|
\end{align}

\begin{tcolorbox}[colback=red!3, colframe=red!60!black, title=注释, breakable]
	由式 \ref{eq:1.53} 有
	\begin{align}
		\hat{H}&=\Omega _{\text{p}}\left| 1 \right> \left< e \right|+\Omega _{\text{s}}\left| 2 \right> \left< e \right|+\text{h.c}.+\Delta \left| e \right> \left< e \right| \notag \\
		&=A\frac{\Omega _{\text{p}}}{A}\left| 1 \right> \left< e \right|+A\frac{\Omega _{\text{s}}}{A}\left| 2 \right> \left< e \right|+\text{h.c}.+\Delta \left| e \right> \left< e \right|  \notag \\
		&=A\left( \frac{\Omega _{\text{p}}}{A}\left| 1 \right> +\frac{\Omega _{\text{s}}}{A}\left| 2 \right> \right) \left< e \right|+\text{h.c}.+\Delta \left| e \right> \left< e \right| \tag{1}
	\end{align}
	定义亮态和暗态
	\begin{align}
		\left| B \right> =\frac{\Omega _{\text{p}}}{A}\left| 1 \right> +\frac{\Omega _{\text{s}}}{A}\left| 2 \right> \tag{2} \\
		\left| D \right> =-\frac{\Omega _{\text{s}}}{A}\left| 1 \right> +\frac{\Omega _{\text{p}}}{A}\left| 2 \right> \tag{3}
	\end{align}
	满足
	\begin{align}
		\left< D \mid B \right> &=\left( -\frac{\Omega _{\text{s}}}{A}\left< 1 \right|+\frac{\Omega _{\text{p}}}{A}\left< 2 \right| \right) \left( \frac{\Omega _{\text{p}}}{A}\left| 1 \right> +\frac{\Omega _{\text{s}}}{A}\left| 2 \right> \right)  \notag \\
		&=-\frac{\Omega _{\text{s}}\Omega _{\text{p}}}{A^2}\left< 1 \mid 1 \right> +\frac{\Omega _{\text{p}}\Omega _{\text{s}}}{A^2}\left< 2\mid 2 \right>  \notag \\
		&=0  \tag{4} \\
		\left< B \mid B \right> &=\left( \frac{\Omega _{\text{p}}}{A}\left< 1 \right|+\frac{\Omega _{\text{s}}}{A}\left< 2 \right| \right) \left( \frac{\Omega _{\text{p}}}{A}\left| 1 \right> +\frac{\Omega _{\text{s}}}{A}\left| 2 \right> \right)  \notag \\
		&=\frac{\Omega _{\text{p}}^{2}}{A^2}+\frac{\Omega _{\text{s}}^{2}}{A^2} \notag \\
		&=1 \tag{5} \\
		\left< D \mid D \right> &=1 \tag{6}
	\end{align}
	于是可以将式 \ref{eq:1.54} 改写为
	 \begin{align}
		\hat{H}=\left( A\left| B \right> \left< e \right|+\text{h.c}. \right) +\Delta \left| e \right> \left< e \right| \tag{7}
	\end{align}
\end{tcolorbox}

其中 \(A(t) = \sqrt{\Omega_\text{p}^2(t) + \Omega_\text{s}^2(t)}\)。现在我们定义“亮”态 \(|B\rangle\) 为:
\begin{align}\label{eq:1.55}
	|B\rangle = \frac{\Omega_\text{p}(t)}{A(t)} |1\rangle + \frac{\Omega_\text{s}(t)}{A(t)} |2\rangle
\end{align}
另一个正交于 \(|B\rangle\) 的“暗”态 \(|D\rangle\) 定义为
\begin{align}\label{eq:1.56}
	|D\rangle = -\frac{\Omega_\text{s}(t)}{A(t)} |1\rangle + \frac{\Omega_\text{p}(t)}{A(t)} |2\rangle
\end{align}
在由 \(\{|1\rangle, |2\rangle\}\) 张开的希尔伯特空间中,只有 \(|B\rangle\) 与 \(|e\rangle\) 有耦合,而 \(|D\rangle\) 态与 \(|e\rangle\) 没有耦合。也就是说,对于任意给定的时间 \(t\),\(\mathcal{H}(t)\) 可以被一个幺正矩阵 \(\mathcal{U}(t)\) 对角化,并且总会有一个瞬时本征态 \(|D(t)\rangle\),其本征能量始终为零。其他两个瞬时本征态将是 \(|B\rangle\) 和 \(|e\rangle\) 的叠加态,其能量为 \(\pm\sqrt{A^2(t) + \Delta^2/4}\)。

\begin{tcolorbox}[colback=red!3, colframe=red!60!black, title=注释, breakable]
	由式 \ref{eq:1.54} 有
	\begin{align}
		H=\underset{\left| D \right> \ \left| B \right> \ \left| e \right>}{\underbrace{\left( \begin{matrix} 	0&		0&		0\\ 	0&		0&		A\\ 	0&		A&		\Delta\\ \end{matrix} \right) }} \tag{1}
	\end{align}
	计算该哈密顿量的本征值
	\begin{align}
		\left| \lambda I-H \right|&=\left| \begin{matrix} 	\lambda&		0&		0\\ 	0&		\lambda&		-A\\ 	0&		-A&		\lambda -\Delta\\ \end{matrix} \right|=\lambda \left[ \lambda \left( \lambda -\Delta \right) -A^2 \right]  \notag \\
		&=\lambda \left[ \lambda ^2-\Delta \lambda -A^2 \right] = 0 \tag{2}
	\end{align}
	解得
	\begin{align}
		\left\{ \begin{array}{l} 	\lambda _1=0\\ 	\lambda _{2,3}=\frac{\Delta \pm \sqrt{\Delta ^2+4A^2}}{2}\\ \end{array} \right. \tag{3}
	\end{align}
\end{tcolorbox}

STIRAP 的关键依赖于这个暗态 \(|D(t)\rangle\)。如果解含时薛定谔方程,当我们旋转到瞬时本征态基矢时,将会有一个额外的项 \(i\hbar(\partial_t \mathcal{U}^\dagger(t))\mathcal{U}(t)\),正如我们在第 \ref{sec:Light Shift} 节中讨论的。这一项的非对角矩阵元可以耦合不同的瞬时本征态。然而,由于这三个瞬时本征态之间总是有有限的能量差,如果 \(\mathcal{H}(t)\) 作为时间的函数变化得足够平滑,\(\mathcal{U}(t)\) 也会变化得足够平滑,因此这些非对角矩阵元可以相对于三个瞬时本征态的能量差来说足够小。因此,在这个绝热情况下,系统在随时间演化过程中可以几乎保持在其中某一个瞬时本征态中。在 STIRAP 的情况下,我们希望系统保持在暗态 \(|D(t)\rangle\) 中。

STIRAP 的思想是利用暗态 \(|D(t)\rangle\) 通过适当地设计 \(\Omega_\text{p}(t)\) 和 \(\Omega_\text{s}(t)\) 的时间依赖性,将原子从 \(|1\rangle\) 转移到 \(|2\rangle\)。为了实现这一目标,在初始时刻,\(\Omega_\text{p}/\Omega_\text{s} \rightarrow 0\),因此 \(|D\rangle \rightarrow |1\rangle\);而在转移的结束时刻,\(\Omega_\text{s}/\Omega_\text{p} \rightarrow 0\),因此 \(|D\rangle \rightarrow |2\rangle\)。这个传态的步骤非常违反直觉。它基本上概括为三步:(1) 最初,原子布居在 \(|1\rangle\) 态,但是首先开启的是 \(|2\rangle\) 和 \(|e\rangle\) 之间的耦合,如图 \ref{fig:1.8}(a1) 所示;(2) 在中间阶段,两束激光的耦合都开启,原子从 \(|1\rangle\) 转移到 \(|2\rangle\),如图 \ref{fig:1.8}(a2) 所示;(3) 当原子逐渐转移到 \(|2\rangle\) 态时,\(|e\rangle\) 和布居较少的 \(|1\rangle\) 之间的耦合仍然保持开启,如图 \ref{fig:1.8}(a3) 所示。我们在图 \ref{fig:1.8}(b) 中示意性地展示了 \(\Omega_\text{s}(t)\) 和 \(\Omega_\text{p}(t)\) 随时间变化的曲线。相应地,\(|1\rangle\) 态和 \(|2\rangle\) 态的布居分别由 \(\Omega_\text{s}^2(t)/A^2(t)\) 和 \(\Omega_\text{p}^2(t)/A^2(t)\) 给出,并绘制在图 \ref{fig:1.8}(c) 中。

\begin{tcolorbox}[colback=red!3, colframe=red!60!black, title=注释, breakable]
	\centering
    \includegraphics[width=0.9\textwidth]{images/chap1-3.png}
\end{tcolorbox}

\begin{figure}[h!]
	\centering
	\includegraphics[width=8cm]{images/1-9.png}  % 指定相对路径
	\caption{产生基态分子的 STIRAP 过程。图中展示了利用一个分子激发态将分子从费什巴赫分子转移到基态分子的 STIRAP 过程。}
	\label{fig:1.9}
\end{figure}

STIRAP 以相干的方式将原子从一个态转移到另一个态,并且这两个态之间的能量差被光子带走,因此没有非弹性能量转移。STIRAP 有两个主要优点,使其非常稳定:(1) 由于这个暗态不涉及激发态 \(|e\rangle\),因此它不受 \(|e\rangle\) 态自发辐射的影响。(2) 该方案对 \(\Omega_\text{p}(t)\) 和 \(\Omega_\text{s}(t)\) 的时间依赖性不敏感,只需要时间曲线满足上述初始和结束条件。
STIRAP 已广泛应用于原子分子物理以及化学领域。在冷原子物理中,STIRAP 主要用于生产超冷基态分子。在这一应用中,它转移的是二体分子态而非单个原子态,但其原理与上述相同。实验始于一份超冷的费什巴赫分子,其能量非常接近散射阈值,尺寸与粒子间距相当。费什巴赫分子将在第 \ref{sec:Feshbach Resonance} 节讨论。我们的目标是将它们转移到基态分子,其能量低于阈值超过 \(10^{14}\) Hz,分子尺寸为数个玻尔半径。这个过程如图 \ref{fig:1.9} 所示。在这种情况下,激发态选择为一个与费什巴赫分子和基态分子都有相当大耦合矩阵元的激发分子态。转移效率可以超过 90\%。由于 STIRAP 的过程是相干的,最终的基态分子仍然保持在非常低的温度。使用这种方法,可以制备出基态分子的简并气体。

\newpage
% 第二章
\section{Two-Body Interaction}\label{sec:Two-Body Interaction}

\subsection{Scattering Length}\label{sec:Scattering Length}
原子之间的相互作用在量子多体物理中起着最重要的作用。超冷原子物理和凝聚态物理的一个主要的共同目标是理解量子多体系统中的相互作用效应。超冷原子物理研究的是稀薄的中性原子气体,而凝聚态物理主要关注固体中的电子气体。这两种系统中多体现象的许多差异可以追溯到它们之间不同的相互作用形式。电子之间的相互作用是库仑排斥作用,而原子之间的相互作用势包括吸引的范德华势和在短距离时的强排斥势。超冷原子之间的相互作用具有以下三个重要特征。

\textbullet\ 
\CJKunderdot{\textit{短程}}(Short Ranged):范德华相互作用 $ V\left( \mathbf{r} \right)  $ 是短程的。当 $ r>r_0 $ 时可以近似认为 $ V\left( \mathbf{r} \right) \approx 0 $,其中 $ r_0 $ 是相互作用的力程。

\textbullet\ 
\CJKunderdot{\textit{稀薄}}(Dilute):超冷原子气体非常稀薄,两个原子之间的平均距离 $d$ 远大于 $r_0$。

\textbullet\ 
\CJKunderdot{\textit{低能}}(Low Energy):超冷原子气体的温度非常低。也就是说,散射态的入射能量 $E = \hbar^2 k^2/(m)$ 与短程势能相比非常小,即 $k r_0 \ll 1$,或者等效地,$\hbar^2 k^2/(m) \ll \hbar^2/(m r_0^2)$。

\begin{tcolorbox}[colback=red!3, colframe=red!60!black, title=注释, breakable]
	\centering
    \includegraphics[width=0.7\textwidth]{images/chap2-1.png}
\end{tcolorbox}

根据前两点,似乎大多数时候,任何两个原子之间的距离都很远,在这个距离下相互作用势为零。因此,从经典角度来看,原子之间不会相互作用,原子气体看起来像是非相互作用气体。然而,从量子力学的角度来看,这个系统不仅是一个相互作用的系统,有时还可以成为一个强相互作用的系统。

\textbf{The Phase Shift. }考虑在相对坐标系中的薛定谔方程:
\begin{equation}\label{eq:2.1}
	\left[ -\frac{\hbar^2}{2\bar{m}}\nabla ^2+V\left( \mathbf{r} \right) \right] \Psi =E\Psi
\end{equation}
其中 $\mathbf{r}$ 代表两个原子的相对坐标,$ \bar{m} $ 是两个原子的约化质量。在这里,我们关注的是 $ V\left( \mathbf{r} \right)  $ 是球对称的情况,并且我们可以在球坐标系中将波函数展开 
\begin{equation}\label{eq:2.2}
	\Psi\left( \mathbf{r} \right) =\sum_{l=0}^{+\infty}{\frac{\chi _{kl}\left( r \right)}{kr}}\mathcal{P}_l\left( \cos \theta \right)
\end{equation}
不同分波的波函数是解耦的。因此可以得到
\begin{equation}\label{eq:2.3}
	\frac{d^2\chi _{kl}}{dr^2}-\frac{l\left( l+1 \right)}{r^2}\chi _{kl}+\frac{2\bar{m}}{\hbar^2}\left( E-V\left( r \right) \right) \chi _{kl}=0
\end{equation}

我们首先考虑$s$-波散射通道,有 $l = 0$。稀薄条件使得,当 $r > r_0$ 时有 $V(r) = 0$。式 \eqref{eq:2.3} 的通解为
\begin{equation}\label{eq:2.4}
	\chi _k=A\sin \left( kr+\delta _k \right)
\end{equation}

\begin{tcolorbox}[colback=red!3, colframe=red!60!black, title=注释, breakable]
	考虑 $s$ 分波和稀薄的条件,式 \eqref{eq:2.3} 化简为
	\begin{equation}
		\frac{d^2\chi _{kl}}{dr^2}+\frac{2\bar{m}}{\hbar^2}E\chi _{kl}=\frac{d^2\chi _{kl}}{dr^2}+k^2\chi _{kl}=0 \tag{1}
	\end{equation}
	上式有两个解 $ \sin \left( kr \right)  $ 和 $ \cos \left( kr \right)  $,一般波函数的叠加系数体现在散射相移中。
\end{tcolorbox}

其中 $\delta _k$ 称为\CJKunderdot{\textit{散射相移}}(phase shift)。散射相移是稀薄量子气体中低能散射的最重要量。由于前面提到的超冷原子气体的稀薄特性,原子几乎没有机会靠的足够近来让我们探讨相互作用势的细节,因此我们只关心在 \( r \gg r_0 \) 区域的波函数。显然,所有的相互作用效应都包含在散射相移 \( \delta_k \) 中。

\begin{tcolorbox}[colback=red!3, colframe=red!60!black, title=注释, breakable]
	\centering
    \includegraphics[width=0.7\textwidth]{images/chap2-2.png}
\end{tcolorbox}

然而,散射相移由波函数在短程的行为决定。为了确定 $\delta_k$,我们需要在 $r < r_0$ 区域的信息。因此我们在 $r = r_0$ 处匹配边界条件:
\begin{equation}\label{eq:2.5}
	\left. \frac{\chi '\left( r>r_0 \right)}{\chi \left( r>r_0 \right)} \right|_{r=r_0}=\frac{k\cos \left( kr_0+\delta _k \right)}{\sin \left( kr_0+\delta _k \right)}\approx \frac{k}{\tan \delta _k}=\left. \frac{\chi '\left( r<r_0 \right)}{\chi \left( r<r_0 \right)} \right|_{r=r_0}
\end{equation}

在约等号后面,我们利用了低能下 $k r_0 \approx 0$ 的性质。现在问题是,为了确定散射相移,我们是否需要知道波函数 $\chi(r < r_0)$ 在 $r < r_0$内的全部信息?让我们考虑这样一种情况:当进入到 $r < r_0$ 的区域时,相互作用势很快掉到很深的地方,在这个区域,入射能量 $E$ 相对于 $ V\left( \mathbf{r} \right)  $ 可以忽略。因此我们可以合理的认为对于低能态,波函数 $\chi(r < r_0)$ 的对能量 $E$ 的依赖可以忽略不计。因此,如果只考虑主阶,我们可以将 $ \chi'(r < r_0)/\chi(r < r_0)|_{r=r_0} $ 简单地视为一个常数,记作 $-1/a_{\text{s}}$,其中 $a_{\text{s}}$ 被称为 \CJKunderdot{\textit{$s$-波散射长度}}(s-wave scattering length)。因此我们有
\begin{equation}\label{eq:2.6}
	\frac{k}{\tan \delta _k}=-\frac{1}{a_{\text{s}}}
\end{equation}

\begin{figure}[h!]
	\centering
	\includegraphics[width=12cm]{images/2-1.jpg}  % 指定相对路径
	\caption{$s$-波散射相移。实线为 $a_{\text{s}} / r_0 = 50$,虚线为 
	$a_{\text{s}} / r_0 = 10^3$,点线为 $a_{\text{s}} / r_0 = -50$,虚点线为 
	$a_{\text{s}} / r_0 = -10^3$。}
	\label{fig:2.1}
\end{figure}

通过式 \eqref{eq:2.6},得到 \( \delta_k \) 与 \( k \) 之间的关系如图\ref{fig:2.1}所示。对于较小的 \( k \),\( \delta_k \) 线性依赖于动量 \( -k a_{\text{s}} \),而对于较大的 \( k \)(但仍相对于 $ 1/r_0 $ 非常小),\( \delta_k \) 会趋向于 \( \pm \pi/2 \)。我们还应该注意到在推导出式 \eqref{eq:2.6} 时,我们仅使用了前面讨论的三个条件,因此式 \eqref{eq:2.6} 对任何 \( a_{\text{s}} \) 的值都有效,包括 \( a_{\text{s}} = \pm \infty \)。当 \( a_{\text{s}} \) 增大时,\( \delta_k \) 的线性区域变得越来越小。当 \( a_{\text{s}} = \pm \infty \) 时,\( \delta_k \) 对于任何非零的 \( k \) 都为 \( \mp \pi/2 \)。

\begin{tcolorbox}[colback=red!3, colframe=red!60!black, title=注释, breakable]
	\centering
    \includegraphics[width=0.7\textwidth]{images/chap2-3.png}
\end{tcolorbox}

我们可以进一步将 \( E \) 作为薛定谔方程在 \( r < r_0 \) 区域中的一个微扰,通过将 \( k / \tan \delta_k \) 展开到 \( k^2 \) ,得到:
\begin{equation}\label{eq:2.7}
	\frac{k}{\tan \delta _k}=-\frac{1}{a_s}+\frac{1}{2}r_{\text{eff}}k^2+\cdot 
	\cdot \cdot
\end{equation}
其前面的系数定义了一个\CJKunderdot{\textit{有效长度}}(effective range) \( r_{\text{eff}} \)。在大多数情况下,\( r_{\text{eff}} \) 项在低能下的贡献可以忽略不计,因此 \( a_s \) 是描述低能两体相互作用的最重要参数。然而,也存在例外情况。例如当 \( a_s \) 接近零时,展开式 \eqref{eq:2.7} 并不是很合适,因为 \( \delta_k \) 不可能对所有 \( k \) 都始终为零。当 \( a_s \to 0 \) 时,可以做如下展开:
\begin{equation}\label{eq:2.8}
	-\frac{\tan \delta _k}{k}=\frac{1}{\frac{1}{a_s}-\frac{1}
	{2}r_{\text{eff}}k^2}\approx a_s+\frac{1}{2}r_{\text{eff}}a_{s}^{2}k^2+\cdot 
	\cdot \cdot
\end{equation}

\begin{tcolorbox}[colback=red!3, colframe=red!60!black, title=注释, breakable]
	由式 \ref{eq:2.7} 有
	\begin{align}
		\frac{\tan \delta _k}{k}&=\frac{1}{-\frac{1}{a_s}+\frac{1}{2}r_{\text{reff}}k^2}=\frac{-a_s}{1-\frac{1}{2}r_{\text{reff}}a_sk^2} \notag \\
		&=-a_s\left( 1+\frac{1}{2}r_{\text{reff}}a_sk^2 \right)  \notag \\
		&=-\left( a_s+\frac{1}{2}r_{\text{reff}}a_{s}^{2}k^2 \right) \notag
	\end{align}
	所以
	\begin{align}
		\tan \delta _k=-\left( ka_s+\frac{1}{2}r_{\text{reff}}a_{s}^{2}k^3 \right) \notag
	\end{align}
	当 $ a_s\rightarrow 0 $ 时,有 $ \tan \delta _k\propto k^3 $。
\end{tcolorbox}

事实上,当 \( a_s \to 0 \) 时,\( r_{\text{eff}} \) 将会发散,使得 \( r_{\text{eff}} a_s^2 \) 保持有限,称 \( v = -r_{\text{eff}} a_s^2 \) 为\CJKunderdot{\textit{散射体积}}(scattering volume)。也就是说,当 \( a_s \) 为有限大小时,\( \tan \delta_k \) 线性依赖于 \( k \),而当 \( a_s \) 趋于零时,线性项也趋于零,\( \tan \delta_k \) 将依赖于 \( k^3 \)。

考虑两个不同的短程势 \( V_1(\mathbf{r}) \) 和 \( V_2(\mathbf{r}) \),例如两种不同原子间的相互作用势,在 \( r < r_0 \) 区域,这两种原子的短程波函数 \( \chi_1(r) \) 和 \( \chi_2(r) \) 是不同的。但是,只要它们在 \( r = r_0 \) 处给出相同的 \( \chi' / \chi \) 值,从而得到相同的散射相移,这两个系统的低能物理就是相同的,尽管短程势的行为截然不同。如果我们进一步关注有效长度可以忽略不计的情况,那么这两个势共享相同的 \( a_s \),并且这个 \( a_s \) 是描述这两个不同短程势的低能物理所需要的唯一参数。这就是所谓的\CJKunderdot{\textit{普适性}}(universality),它表明具有完全不同微观细节的系统,可以通过少数几个参数来普遍描述。

\( s \)-波散射长度也具有明确的几何意义。在零能极限下,\( r > r_0 \) 区域的 \( s \)-波波函数可以展开为:
\begin{equation}\label{eq:2.9}
	\chi \left( r \right) \propto \sin \left( kr+\delta _k \right) \approx \sin 
	\delta _k+\cos \delta _k\left( kr \right)  $$ $$ \propto 1+\frac{k}{\tan 
	\delta _k}r=1-\frac{r}{a_s}
\end{equation}
显然,\( \chi(r = a_s) = 0 \);也就是说,\( a_s \) 是零能径向波函数的节点。

下面考虑一个有限深方势阱模型(toy model):\( V(r) = -V_0 \) (\( V_0 > 0 \)) 当 \( 0 < r < r_0 \), \( V(r) = 0 \) 当 \( r > r_0 \),在 $ r 0 $ 处我们考虑\CJKunderdot{\textit{刚性边界条件}}(hard core boundary condition)。这个简单的玩具模型模拟了真实的原子间相互作用势。在这个模型中,$0 < r < r_0$ 区域的零能波函数由下式给出: 
\begin{equation}\label{eq:2.10}
	\chi \left( r \right) =\sin \left( \sqrt{\frac{2V_0\bar{m}}{\hbar^2}}r \right)
\end{equation}
上式在 \( r = 0 \) 处满足无限高势垒边界条件,并且在 \( r = r_0 \) 处的斜率决定了 $ r_0 $ 以外的波函数,其节点又决定了 \( a_s \)。通过这种方式,容易看出随着势阱深度 \( V_0 \) 的增加,\( a_s \) 是如何变化的。当吸引势较浅且 $ \sqrt{2V_0\bar{m}/\hbar^2}r_0<\pi /2 $ 时,如图 \ref{fig:2.2}(a) 左图所示,波函数的节点出现在负半轴,使得 \( a_s \) 为负值。随着 \( V_0 \) 的增加,当 $ \sqrt{2V_0\bar{m}/\hbar^2}r_0 $ 接近 $ \pi /2 $ 时,零能波函数的斜率接近于零。因此,\( a_s \) 先接近于 \( -\infty \),然后从 \( -\infty \) 跳变到 \( +\infty \),如图 \ref{fig:2.2}(a) 中间所示。在这个跳变处,散射相移也从 \( \pi /2 \) 跳变到 \( -\pi /2 \)。然后,当 \( V_0 \) 进一步增加,斜率变为负值,节点来到了正值,如图 \ref{fig:2.2}(a) 右图所示。随着 \( V_0 \) 进一步增大,\( a_s \) 从 \( +\infty \) 减小到有限的正值。这个例子表明,\( a_s \) 可以取从 \( -\infty \) 到 \( +\infty \) 的任何值,如图 \ref{fig:2.2}(b) 所示。
\begin{figure}[h!]
	\centering
	\includegraphics[width=12cm]{images/2-2.jpg}  % 指定相对路径
	\caption{散射长度的几何意义。(a) 方势阱模型中 \( s \)-波散射长度的几何意义。(b) \( s \)-波散射长度作为势阱深度的函数。低能束缚态的能量也画了出来。}
	\label{fig:2.2}
\end{figure}
	
在这个模型中,也可以得到:
\begin{equation}\label{eq:2.11}
	-\frac{1}{a_s}=\left. \frac{\chi '\left( r<r_0 \right)}{\chi \left( r<r_0 \right)} \right|_{r=r_0}=\frac{\sqrt{\frac{2V_0\bar{m}}{\hbar^2}}}{\tan \left( \sqrt{\frac{2V_0\bar{m}}{\hbar^2}}r_0 \right)}
\end{equation}
显然,随着 \( V_0 \) 的增加,\( a_s \) 会反复从 \( -\infty \) 变到 \( +\infty \)。也就是说,将会有一系列不同的 \( V_0 \) 值给出相同的 \( a_s \) 值。结合我们上面关于普适性的讨论,这意味着这些不同 \( V_0 \) 值对应的低能物理是相同的。

值得一提的是,散射相移的讨论可以推广到其他更高阶的分波。可以证明,对于第 \( l \) 阶分波,相应的散射相移 \( \delta_k \propto k^{2l+1} \)。因此,对于低能散射,高阶分波的散射相移相比于 \( s \)-波将受到压制。这意味着对于低能散射只要 \( s \)-波散射通道不被禁止,相互作用效应就主要由 \( s \)-波散射通道主导。

\textbf{The Shallow Bound State. }上面我们已经考虑了能量 \( E > 0 \) 的低能散射态,现在我们转而考虑能量 \( E < 0 \) 的束缚态。散射态和束缚态的区别体现在它们在大 \( r \) 处的渐近行为。对于散射态,其波函数在大 \( r \) 处保持以固定动量的振荡。散射态的能谱是连续的,短程边界条件决定了散射相移。但对于束缚态,其波函数在大 \( r \) 处指数衰减;能量是离散的,且由短程边界条件决定。明确地说,在 \( r > r_0 \) 范围内,径向波函数表示为
\begin{equation}\label{eq:2.12}
	\chi =Ae^{-r\sqrt{2\bar{m}\left| E \right|/\hbar^2}}
\end{equation}
类似于对低能散射态的讨论,这里我们关注的是束缚态能量的绝对值远小于势阱的深度,以至于在短程 \( r < r_0 \) 处的波函数与束缚态能量无关。因此,我们可以在 \( r = r_0 \) 处对这个束缚态波函数匹配边界条件,得到
\begin{equation}\label{eq:2.13}
	\left. \frac{\chi '}{\chi} \right|_{r=r_0}=-\sqrt{\frac{2\bar{m}\left| E \right|}{\hbar^2}}=-\frac{1}{a_s}
\end{equation}
显然,如果 \( a_s < 0 \),则式 \eqref{eq:2.13} 没有解,这意味着对于负的 \( a_s \) 不存在\CJKunderdot{\textit{浅束缚态}}(shallow bound state)。但当 \( a_s > 0 \) 时,我们有一个束缚态解:(画在图 \ref{fig:2.2}(b)中)
\begin{equation}\label{eq:2.14}
	E_{\text{b}}=-\frac{\hbar^2}{2\bar{m}a_{s}^{2}}
\end{equation}

然而,我们必须非常小心地对待关于负 \( a_s \) 不存在束缚态的说法。实际上,如图 \ref{fig:2.2} 所示,第一个束缚态出现在 $ \sqrt{2V_0\bar{m}/\hbar^2}r_0=\pi /2 $ 处,并且当 \( V_0 \) 进一步增加时它不会消失。然而,当 \( V_0 \) 继续增大到一定值后,\( a_s \) 将重新变为负数。也就是说,对于这些负的 \( a_s \),实际上是存在束缚态的。为了将这一事实与式 \eqref{eq:2.13} 相协调,我们应当注意到式 \eqref{eq:2.13} 仅适用于低能束缚态,因为我们假设了在短程波函数与能量无关。式 \eqref{eq:2.13} 无解只排除了低能束缚态存在的可能性,但并不排除深束缚态存在的可能。实际上,可以很容易看到,当 \( a_s \) 再次变为负数时,一开始在 $ \sqrt{2V_0\bar{m}/\hbar^2}r_0=\pi /2 $ 处出现的束缚态变成了一个足够深的束缚态。此外,即使对于正的 \( a_s \),式 \eqref{eq:2.14} 仅在 \( a_s/r_0 \gg 1 \) 时成立。当束缚态足够深时,束缚能将偏离式 \eqref{eq:2.14},短程的细节将变得重要。

\textbf{Features of a Scattering Resonance. }从上述讨论中,我们还可以发现以下三件事情将会同时发生,这就是所谓的\CJKunderdot{\textit{ \( s \)-波散射共振}}(s-wave scattering resonance):

1. 散射长度从 $-\infty$ 跳跃到 $+\infty$。

2. 散射相移跳跃 $\pi$。

3. 有一个束缚态出现在零能。

现在我们已经介绍了散射长度 \( a_{\text{s}} \) 作为 \( s \)-波散射的核心概念。下面我们将讨论两个解释 \( a_{\text{s}} \) 物理意义的重要问题。

\textbf{In What Sense Does a Larger $ \left| a_{\text{s}} \right| $ Mean a Stronger Interaction? }我们已经从薛定谔方程的本征态角度考虑了两体问题。现在我们引入另一种观点。考虑一个沿 \( \hat{z} \) 方向的入射波 \( e^{ikz} \),它被散射到各个方向,在远处的波函数可以写为:
\begin{equation}\label{eq:2.15}
	\Psi =e^{ikz}+f\left( \theta \right) \frac{e^{ikr}}{r}
\end{equation}
其中 \( f(\theta) \) 被称为\CJKunderdot{\textit{散射振幅}}(scattering amplitude)。为了确定 \( f(\theta) \),我们需要先将式 \eqref{eq:2.15} 重写为:
\begin{equation}\label{eq:2.16}
	\Psi =\frac{1}{2ikr}\left[ \sum_{l=0}^{+\infty}{\left( 2l+1 \right) \mathcal{P}_l\left( \cos \theta \right) \left( e^{i\left( kr-l\pi /2 \right)}-e^{-i\left( kr-l\pi /2 \right)} \right)} \right] +f\left( \theta \right) \frac{e^{ikr}}{r}
\end{equation}
两种观点应该给出相同的结果。通过将上式与式 \eqref{eq:2.2} 和式 \eqref{eq:2.4} 进行比较,并考虑 \( l = 0 \) 的散射通道,可以得到 \( s \)-波散射振幅为:
\begin{equation}\label{eq:2.17}
	f_s\left( \theta \right) =\frac{e^{2i\delta}-1}{2ik}=-\frac{1}{ik-k/\tan \delta}=-\frac{1}{1/a_s+ik}
\end{equation}

\begin{tcolorbox}[colback=red!3, colframe=red!60!black, title=注释, breakable]
	由式 \eqref{eq:2.2} 和式 \eqref{eq:2.4} 有
	\begin{align}
		\Psi &=\frac{\sin \left( kr+\delta _k \right)}{kr} \notag \\
		&=\frac{e^{ikr+i\delta _k}-e^{-ikr-i\delta _k}}{2ikr} \notag \\
		&\propto \frac{e^{ikr+i2\delta _k}-e^{-ikr}}{2ikr}  \tag{1}
	\end{align}
	由式 \eqref{eq:2.16} 有
	\begin{align}
		\Psi &=\frac{1}{2ikr}\left( e^{ikr}-e^{-ikr} \right) +f_{\text{s}}\frac{e^{ikr}}{r} \notag \\
		&=\frac{\left( 2ikf_{\text{s}}+1 \right) e^{ikr}-e^{-ikr}}{2ikr} \tag{2}
	\end{align}
	比较上面两式,有 $ 2ikf_{\text{s}}+1=e^{2i\delta _k} $,所以
	\begin{align}
		f_{\text{s}}&=\frac{e^{2i\delta _k}-1}{2ik} \notag \\
		&=\frac{\cos \left( 2\delta _k \right) +i\sin \left( 2\delta _k \right) -1}{2ik} \notag \\
		&=\frac{-2\sin ^2\delta _k+i2\sin \delta _k\cos \delta _k}{2ik} \notag \\
		&=-\frac{\sin \delta _k\left( \sin \delta _k-i\cos \delta _k \right)}{ik} \notag \\
		&=-\frac{\sin \delta _k}{ik\left( \sin \delta _k+i\cos \delta _k \right)} \notag \\
		&=-\frac{1}{ik\left( 1+\frac{i}{\tan \delta _k} \right)} \notag \\
		&=-\frac{1}{ik+\frac{1}{a_{\text{s}}}} \tag{3}
	\end{align}
	得到式 \eqref{eq:2.17}。
\end{tcolorbox}

上式与 \( \theta \) 无关。对于全同玻色子,\CJKunderdot{\textit{散射截面}}(scattering cross section ) \( \sigma \) 由 \( 8\pi |f_s|^2 \) 给出。如果 \( |k a_s| \ll 1 \),我们有 \( f(\theta) \sim -a_s \),故散射截面为 \( \sigma = 8\pi a_s^2 \)。因此 \( |a_s| \) 越大,散射截面也就越大。从这个意义上,可以说 \( a_s \) 的绝对值代表了相互作用的强度(the strength of the interaction)。但这个论断不能推广到 \( a_s \) 非常大甚至是发散的情况,因为如果 \( |k a_s| \gg 1 \),则 \( f(\theta) \) 应该近似为 \( -1/(ik) \),此时散射截面变为 \( 8\pi/k^2 \)。有趣的是,在这个区域内,散射截面主要依赖于碰撞粒子的动量,而不依赖于其他任何参数。这就是所谓的\CJKunderdot{\textit{幺正区域}}(unitary regime)。正如散射截面的公式所示,在幺正区域中多体系统的相互作用能量仅依赖于密度和温度。这些现象将在第 \ref{sec:The Fermi Liquid} 章和第 \ref{sec:The Fermi Superfluid} 章中详细讨论。

\begin{tcolorbox}[colback=red!3, colframe=red!60!black, title=注释, breakable]
	远离共振时:散射长度增大,相互作用变强。
\end{tcolorbox}

\textbf{In What Sense Does $ a_{\text{s}}>0 $ Mean Repulsive Interaction? }当我们讨论短程排斥相互作用时,通常我们自然想到的是一个\CJKunderdot{\textit{钢球模型}}(hard core potential),大小为 \( R_0 \)。也就是说,\( V(r \leq R_0) = +\infty \),\( V(r > R_0) = 0 \),这个排斥势使得波函数在 \( r = R_0 \) 处为零,因此可以得到 \( \delta_k = -kR_0 \)。我们在这里考虑的原子间相互作用是一个吸引势,其微观细节与钢球模型是截然不同的。然而,由式 \eqref{eq:2.6} 给出的散射相移的低能展开($ \delta _k=-ka_{\text{s}} $)可以与 \( \delta_k = -kR_0 \) 在形式上一致。因此,就散射态的低能散射相移而言,正的 \( a_s \) 等价于一个钢球排斥相互作用。换句话说,由于稀薄气体中的原子只能感受到一个散射相移 \( \delta_k \),因此对于足够低能的原子,它们无法区分真实的原子间相互作用与钢球相互作用。

\begin{tcolorbox}[colback=red!3, colframe=red!60!black, title=注释, breakable]
	大于零的 \( a_s \)($ \ll d $)代表排斥相互作用
\end{tcolorbox}

然而,我们要强调的是,这种等价性仅在低能散射态和很小的 \( a_s \) (\( k a_s \ll 1 \))中成立。这里有几个原因。首先,在大动量下,当 \( k a_s \) 较大时,由式 \eqref{eq:2.6} 给出的散射相移将会饱和到 \( -\pi/2 \),而钢球模型的散射相移则继续线性增大。其次,为了使钢球模型在原子气体中有效,钢球的半径 \( R_0 \) 应该远小于粒子间距,通常为 \( 1/k \)。这也要求 \( a_s \sim R_0 \ll 1/k \)。最后,如上所讨论的,对于很大的 \( a_s \),存在一个低能束缚态,而在钢球势中没有这样的低能束缚态。总之,只有在以下两种情况下,正的 \( a_s \) 可以被视为排斥相互作用:

1. \( k a_s \ll 1 \),并且只考虑低能散射态。

2. 束缚态足够深,以至于可以在低能散射中忽略不计。

\begin{tcolorbox}[colback=red!3, colframe=red!60!black, title=注释, breakable]
	\centering
    \includegraphics[width=15cm]{images/chap2-4.png}
\end{tcolorbox}

\textbf{How Short Range Is Short Ranged? }到目前为止,我们考虑的都是一个有限力程的相互作用势,这个相互作用在 \( r_0 \) 以外被严格的视为零。但真实的范德华势在远距离处是以幂函数衰减的。现在我们将回过头来简要地回顾一下,什么情况下可以用一个严格的短程势来替代真实的相互作用势。
	
让我们再次回顾一下如何求解短程势的波函数。在 \( r > r_0 \) 范围内,势能项为零,哈密顿量仅包含动能项。对于三维动能算符,\( s \)-波通道有两个独立的解,可以取为 \( \sin(kr)/(kr) \) 和 \( \cos(kr)/(kr) \)。在远距离处,一般的波函数是这两个解的叠加,叠加的相位给出了散射相移,而散射相移应该由短程的物理过程决定。
这个方法可以直接推广到幂函数衰减势。唯一的不同是需要找到与 \( 1/r^\alpha \) 势相对应的解。事实证明,对于 \( s \)-波情况,也可以写出两个独立的解,它们的渐近解分别为 \( \sin(kr)/(kr) \) 和 \( \cos(kr)/(kr) \)。因此,一般的波函数是这两个解的叠加,叠加系数决定了散射相移,而散射相移是由短程的物理过程决定的。

类似的处理可以推广到更高阶的分波。通过这种方式,可以证明对于第 \( l \) 分波,\( \tan \delta_k \propto k^{2l+1} \) 如果 \( 2l + 1 \leq \alpha - 2 \), \( \tan \delta_k \propto k^{\alpha - 2} \) 如果 \( 2l + 1 \geq \alpha - 2 \)。因此,就低能物理而言,如果要将长程势等效于短程势,则 \( \alpha - 2 \) 必须大于 \( 2l + 1 \)。因此,对于 \( s \)-波通道,\( \alpha \) 应该大于 3。对于实际的范德瓦尔斯势,\( \alpha = 6 \),这意味着对于 \( l = 0, 1 \) 的散射通道,可以通过短程势来近似处理。

\subsection{Zero-Range Models}\label{sec:Zero-Range Models}
我们已经讨论了用一个有限力程的相互作用势来研究两体问题的低能物理,并且证明了在大多数情况下,\( s \)-波散射长度 \( a_s \) 是描述稀薄量子气体中低能相互作用所需的唯一参数。现在我们希望建立一个有效模型来描述多体系统中的相互作用效应,并且我们希望这个有效模型具有以下两个性质:

\textbullet\ 
受两体讨论的启发,我们希望在有效模型中 \( s \)-波散射长度 \( a_s \) 仍是唯一的参数,并忽略原子间相互作用势的微观细节。

\textbullet\ 
为了方便后续多体理论的讨论,我们将采用零力程模型,也就是说,两个原子仅当它们在空间同一点时才发生相互作用。在第 \ref{sec:Scattering Length} 节中,我们假设相互作用的范围 \( r_0 \) 是一个有限大小的数,在本节中,我们令 \( r_0 \) 趋于零。

在这里,我们所说的\CJKunderdot{\textit{有效}}(effective)是指低能散射性质,包括低能散射态的散射相移和浅束缚态能量,都可以在这个有效模型很好地展现出来。下面将介绍两种能够达成这一目标的有效模型。

\textbf{Pseudopotential. }零力程模型的最简单形式是一个 $ \delta $ 函数势 \( V(r) \propto \delta(r) \)。显然,对于 \( r \neq 0 \),有 \( V(r) = 0 \),\( \chi(r) = \sin(kr + \delta_k) \) 总是能够满足薛定谔方程。问题是当 \( r \to 0 \) 时,薛定谔方程是否仍被满足。然而,如式 \ref{eq:2.9} 所示,在零能极限下,$ \chi \left( r \right) \propto 1-r/a_{\text{s}} $,因此 $ \Psi \left( r \right) \propto 1/r-1/a_{\text{s}} $,在短程会按 \( 1/r \) 发散。因此,只有 $ \delta $ 函数势会导致能量发散。我们注意到,这种 \( 1/r \) 的发散并不具有物理意义,因为在上面讨论的有限力程模型中,自由波函数在 \( r_0 \) 处被截断,短程波函数需由微观势确定。因此,一个正确定义的相互作用势应该能够在作用 \( \delta \)-函数相互作用之前,消除 \( r \to 0 \) 时的 \( 1/r \) 发散。

我们将这个势记作 $V(r) = \delta(r) \hat{O}(r)$,并且 $V(r)$ 应该满足薛定谔方程:
\begin{equation}\label{eq:2.18}
	\left[ -\frac{\hbar^2}{2\bar{m}}\nabla ^2+V\left( \mathbf{r} \right) \right] \Psi =E\Psi
\end{equation}
其中 \( \Psi(r) = \sin(kr + \delta_k)/kr \)。为了分析 \( r \to 0 \) 的极限,我们再次考虑波函数 $ \Psi \left( r \right)  $ 在 \( r = 0 \) 附近的展开
\begin{equation}\label{eq:2.19}
	\Psi \left( r \right) =\frac{1}{r}-\frac{1}{a_s}+o\left( kr \right)
\end{equation}
很容易证明:
\begin{equation}\label{eq:2.20}
	\left[ \partial _{\text{r}}\cdot \text{r} \right] \Psi \left( r \right) =-\frac{1}{a_s}+o\left( kr \right)
\end{equation}

\begin{tcolorbox}[colback=red!3, colframe=red!60!black, title=注释, breakable]
	由式 \ref{eq:2.20} 有
	\begin{align}
		\left( \partial _r\cdot r \right) \left( \frac{1}{r}-\frac{1}{a_{\text{s}}}+\cdots \right) =-\frac{1}{a_{\text{s}}}+\cdots  \tag{1}
	\end{align}
\end{tcolorbox}

从而消除了 $1/r$ 在短程的发散。因此,当 $\hat{O}(r)$ 取为以下形式时
\begin{equation}\label{eq:2.21}
	\hat{O}\left( r \right) =\frac{2\pi \hbar^2a_s}{\bar{m}}\partial _r\cdot r
\end{equation}
我们有
\begin{equation}\label{eq:2.22}
	V\left( \mathbf{r} \right) \Psi \left( r \right) =-\frac{2\pi \hbar^2}{\bar{m}}\delta \left( \mathbf{r} \right)
\end{equation}

\begin{tcolorbox}[colback=red!3, colframe=red!60!black, title=注释, breakable]
	由式 \ref{eq:2.21} 有
	\begin{align}
		V\left( r \right) \Psi &=\left[ \frac{2\pi \hbar^2a_{\text{s}}}{\bar{m}}\delta \left( r \right) \left( \partial _r\cdot r \right) \right] \left( \frac{1}{r}-\frac{1}{a_{\text{s}}}+\cdots \right)  \notag \\
		&=\frac{2\pi \hbar^2a_{\text{s}}}{\bar{m}}\delta \left( r \right) \left( -\frac{1}{a_{\text{s}}}+\cdots \right)  \notag \\
		&=-\frac{2\pi \hbar^2}{\bar{m}}\delta \left( r \right)  \tag{1}
	\end{align}
\end{tcolorbox}

因为在三维空间中有:
\begin{equation}\label{eq:2.23}
	-\frac{\hbar^2}{2\bar{m}}\nabla ^2\Psi \left( r \right) =\frac{2\pi \hbar^2}{\bar{m}}\delta \left( \mathbf{r} \right)
\end{equation}

\begin{tcolorbox}[colback=red!3, colframe=red!60!black, title=注释, breakable]
	\begin{align}
		\nabla^2\left(\frac{1}{r}\right) \propto \delta \left( r \right) \tag{1}
	\end{align}
\end{tcolorbox}

因此零能波函数的薛定谔方程可以被满足。这个相互作用势被称为\CJKunderdot{\textit{费米赝势}}(Fermi's pseudo-potential)。

\textbf{Renomalizable Contact Potential. }赝势模型可以很好地展现低能物理。然而,它有一个缺点,那就是算符不是厄米的。因此,在许多情况下,使用赝势并不十分方便,尤其是在需要使用多体哈密顿量的二次量子化形式时。对于研究多体物理,使用 $ \delta $ 函数势,如 $V(\mathbf{r}) = g\delta(\mathbf{r})$,仍然是方便的。虽然我们已经知道它会在短程造成发散问题,但我们仍然继续推进,看看这个问题有多严重,以及是否有方法来解决这个问题。

在这里,我们考虑具有 $ \delta $ 函数相互作用势的自旋-1/2 费米子的情况。对于 $ \delta $ 函数势,自旋-1/2 费米子的二次量子化哈密顿量可以写作
\begin{equation}\label{eq:2.24}
	\hat{\mathcal{H}}=\int{d^3\mathbf{r}}\left( \sum_{\sigma}{\Psi _{\sigma}^{\dag}\left( \mathbf{r} \right) \left( -\frac{\hbar^2}{2m}\nabla ^2 \right) \Psi _{\sigma}\left( \mathbf{r} \right)}+g\Psi _{\uparrow}^{\dag}\left( \mathbf{r} \right) \Psi _{\downarrow}^{\dag}\left( \mathbf{r} \right) \Psi _{\downarrow}\left( \mathbf{r} \right) \Psi _{\uparrow}\left( \mathbf{r} \right) \right)
\end{equation}
其中 \(\Psi^\dagger_\sigma(\mathbf{r})\) 和 \(\Psi_\sigma(\mathbf{r})\) (\(\sigma = \uparrow, \downarrow\)) 是位置 $\mathbf{r}$ 处费米子的产生算符和湮灭算符。在动量空间中,该哈密顿量表示为
\begin{equation}\label{eq:2.25}
	\hat{\mathcal{H}}=\sum_{\mathbf{k}\sigma}{\frac{\hbar^2\mathbf{k}^2}{2m}\Psi _{\mathbf{k}\sigma}^{\dag}}\Psi _{\mathbf{k}\sigma}+\frac{g}{V}\sum_{\mathbf{k,k}_1,\mathbf{k}_2}{\Psi _{\frac{\mathbf{k}}{2}+\mathbf{k}_1,\uparrow}^{\dag}}\Psi _{\frac{\mathbf{k}}{2}-\mathbf{k}_1,\downarrow}^{\dag}\Psi _{\frac{\mathbf{k}}{2}-\mathbf{k}_2,\downarrow}\Psi _{\frac{\mathbf{k}}{2}+\mathbf{k}_2,\uparrow}
\end{equation}
其中 \( V \) 是系统的体积。第二项表示原子之间的散射,其中质心动量 $ \mathbf{k} $ 是守恒的,而相对动量从 $ \mathbf{k}_2 $ 变化到 $ \mathbf{k}_1 $。

\begin{tcolorbox}[colback=red!3, colframe=red!60!black, title=注释, breakable]
	不论动量转移有多大,概率是一样的。
\end{tcolorbox}

我们首先使用式\eqref{eq:2.25}的哈密顿量计算两体散射 T 矩阵。我们考虑一个满足能量守恒的散射过程,其中入射态和出射态的能量均为 \( E \),且质心动量为零。由于相互作用顶点 \( g \) 是一个与动量无关的常数,主导阶数的图是从入射态直接散射到出射态,其贡献为 \( g \),如图 2.3(a) 所示。下一个阶数的图涉及中间态,中间态的相对动量 \( p \) 可以取任何值。它的贡献可以通过二阶过程计算得到:
\begin{equation}\label{eq:2.26}
	\frac{1}{V}\sum_{\mathbf{p}}{g\frac{1}{E-\frac{\hbar^2\mathbf{p}^2}{m}+i0^+}g}
\end{equation}
其中 $i0^+$是积分计算所需的数学技术细节,也是因果关系的结果。此外,可以通过引入更多中间态来系统地考虑所有高阶贡献,如图\ref{fig:2.3}(a)所示的所谓阶梯图所说明的那样。结果表明,对于二体问题,与后面章节将讨论的多体情况不同,阶梯图的求和是一个精确解。阶梯图的求和导出了所谓的施温格-戴森方程(Schwinger–Dyson equation ):
\begin{align}\label{eq:2.27}
	T_2\left( E \right) &=g+\frac{1}{V}\sum_{\mathbf{p}}{g\frac{1}{E-\frac{\hbar^2\mathbf{p}^2}{m}+i0^+}g}+\cdots \notag\\
	&=g+\frac{g}{V}\sum_{\mathbf{p}}{\frac{1}{E-\frac{\hbar^2\mathbf{p}^2}{m}+i0^+}T_2\left( E \right)}
\end{align}
因此得到
\begin{equation}\label{eq:2.28}
	T_2\left( E \right) =\frac{g}{1-\frac{g}{V}\sum_{\mathbf{p}}{\frac{1}{E-\frac{\hbar^2\mathbf{p}^2}{m}+i0^+}}}
\end{equation}

\begin{figure}[h!]
	\centering
	\includegraphics[width=14cm]{images/2-3.jpg}  % 指定相对路径
	\caption{两体散射的 T 矩阵。两体 T 矩阵的阶梯图 (a) 是可重整化接触势模型\eqref{eq:2.24}的图,(b) 是双通道模型\eqref{eq:2.66}的图。}
	\label{fig:2.3}
\end{figure}

这里重要的是注意到式\eqref{eq:2.28}中的动量求和在大动量时表现为 \( \int d^3p (1/p^2) \),在三维中会在大动量时发散。这个发散来自能量积分的上限,被称为紫外发散(ultraviolet divergence)。这种紫外发散意味着短程物理没有得到正确处理。在这里,这意味着自由波函数的短程 \( 1/r \) 行为不应被延拓到 \( r \to 0 \) 的极限,并且 δ 函数接触势是不合适的。

这个发散问题也可以从动量空间中的哈密顿量式\eqref{eq:2.25}来观察,其中散射顶点被认为与动量转移无关,因为 δ 函数势的傅里叶变换是一个常数。然而,这是不物理的,因为在任何具有有限范围 \( r_0 \) 的物理模型中,当转移的动量远大于 \( \hbar / r_0 \) 时,这个散射顶点总是趋向于零。通过考虑散射顶点的动量依赖性,可以避免式\eqref{eq:2.28}中求和时的大动量发散。然而,散射顶点在大动量下的动量依赖性来源于微观势的短程结构,这是我们不想包括的非普适物理。

因此,我们遇到了一个两难的局面。一方面,我们知道零范围的 \( \delta \)-函数势,或者等效地说,在大动量下动量无关的散射顶点,是不物理的,这会导致紫外发散。另一方面,短程势的细节,或者在大动量下散射顶点的动量依赖性,是非普适的,我们不想显式地包含它。为了克服这个问题,我们将采用重正化(renormalization)的思想。我们仍然使用 \( \delta \)-函数势,但我们不会将相互作用参数 \( g \) 作为物理参数来处理。我们应该找到一种方法,正确地重正化相互作用参数 \( g \),并将其与物理参数 \( a_s \) 关联起来。因此,让我们重写:
\begin{align}\label{eq:2.29}
	T_2\left( E \right) &=\frac{g}{1-\frac{g}{V}\sum_{\mathbf{p}}{\frac{1}{E-\hbar^2\mathbf{p}^2/\left( m \right) +i0^+}}} \notag\\
	&=\frac{1}{\frac{1}{g}+\frac{1}{V}\sum_{\mathbf{p}}{\frac{1}{\hbar^2\mathbf{p}^2/\left( m \right)}}-\frac{1}{V}\sum_{\mathbf{p}}{\left( \frac{1}{E-\hbar^2\mathbf{p}^2/\left( m \right) +i0^+}+\frac{1}{\hbar^2\mathbf{p}^2/\left( m \right)} \right)}} \notag\\
	&=\frac{1}{\frac{1}{g}+\frac{1}{V}\sum_{\mathbf{p}}{\frac{1}{\hbar^2\mathbf{p}^2/\left( m \right)}}+\frac{ikm}{4\pi \hbar^2}}
\end{align}
其中 $ k=\sqrt{mE/\hbar^2} $。这个两体 \( T \)-矩阵应该与上面两体计算确定的式\eqref{eq:2.17}中的 \( s \)-波散射振幅相关;因此,我们有:
\begin{equation}\label{eq:2.30}
	T_2\left( E \right) =\frac{4\pi \hbar^2}{m}\frac{1}{\frac{1}{a_{\text{s}}}+ik}=\frac{1}{\frac{1}{g}+\frac{1}{V}\sum_{\mathbf{p}}{\frac{1}{\hbar^2\mathbf{p}^2/\left( m \right)}}+\frac{ikm}{4\pi \hbar^2}}
\end{equation}
因此,我们得到了一个重要的重正化恒等式,它将 \( g \) 与物理量 \( a_s \) 关联起来,即:
\begin{equation}\label{eq:2.31}
	\frac{m}{4\pi \hbar^2a_{\text{s}}}=\frac{1}{g}+\frac{1}{V}\sum_{\mathbf{p}}{\frac{1}{\hbar^2\mathbf{p}^2/m}}
\end{equation}

总结来说,我们将使用式\eqref{eq:2.24}或\eqref{eq:2.25}作为多体系统的模型。然而,在使用这个模型时,常常会遇到紫外发散的问题。当遇到紫外发散时,我们应该使用式\eqref{eq:2.31},将 \( g \) 替换为物理参数 \( a_s \),并且这个替换同时消除了发散。这个重要的结果将在后续章节中反复使用。

然而,存在一个重要的问题。我们通过匹配两体散射幅度来获得重正化条件。那么我们如何确保这个重正化条件对于多于两个粒子的系统也能适用呢?一般来说,答案是它可能适用,也可能不适用。如果适用,则该理论被称为可重正化的(renormalizable);如果不适用,则意味着在少体或多体系统中出现了额外的高能尺度,这些能量尺度是重要的。实际上,正如我们将在第五章和第六章中看到的,关于自旋 \( \frac{1}{2} \) 费米子的理论是可重正化的。但对于无自旋玻色子,重正化条件实际上是不可行的。从第二章第六节可以看出,我们将讨论玻色子的三体问题。我们将看到,为了使能谱从下界有界,需要引入一个额外的高能截止尺度。

\subsection{Spin-Dependent Interaction}\label{sec:Spin-Dependent Interaction}
在上述讨论中,我们并没有明确考虑碰撞中原子的自旋自由度的作用。实际上,原子的自旋在两体碰撞中可以发挥非常重要的作用,其作用在零磁场极限下和有限大小磁场范围内是不同的。在零磁场极限下,自旋的旋转对称性得到保持,而自旋旋转对称性对两体相互作用的形式施加了约束,本节将对此进行讨论。在磁场为有限大小的范围内,自旋旋转对称性由于塞曼能量的存在而被破坏,但自旋的塞曼能量可以作为调节两体相互作用的工具,这将在 \ref{sec:Feshbach Resonance} 节中讨论。

\textbf{Alkali-Metal Atoms. }我们首先考虑两个自旋为 \( f \) 的碱金属原子之间的碰撞(在这里,我们用小写字母 \( f \) 来表示单个原子的自旋,用大写字母 \( F \) 来表示两个原子的总自旋。)。这里的自旋指的是总的超精细自旋。简单起见,我们以自旋为 \( f = 1 \) 的玻色子为例,例如 \( ^{87}\text{Rb} \) 和 \( ^{23}\text{Na} \) 原子的基态。由于自旋旋转对称性,碰撞中的两个原子的总自旋 \( F \) 应该是守恒的,因此在 \( f = 1 \) 的情况下,两个原子的总自旋可以是 0、1 或 2。因此,相互作用势可以在总自旋基矢中以对角形式表示
\begin{equation}\label{eq:2.32}
	\hat{V}\left( \mathbf{r} \right) =\frac{2\pi \hbar^2}{\bar{m}}\left( a_0\mathcal{P}_0+a_2\mathcal{P}_2 \right) \delta \left( \mathbf{r} \right) \partial _r\cdot r
\end{equation}
其中 \( a_0 \) 和 \( a_2 \) 分别表示在 \( F = 0 \) 和 \( F = 2 \) 通道中的散射长度。这里 \( F = 1 \) 通道不参与 s-波散射,因为对于总自旋 \( F = 1 \) 的情况,自旋波函数是反对称的,因此空间波函数也必须是反对称的,以使得总波函数为对称。这意味着在该通道中,s-波散射是被禁止的。

投影算符 $ \mathcal{P}_{\text{F}} $ 用于将两个原子的自旋波函数投影到总自旋为 \( F \) 的子空间中。为了更明确地写出 $ \mathcal{P}_{\text{F}} $,即以可观测量的形式表示,我们可以利用以下两个恒等式。首先,根据定义,单位算符可以写为
\begin{equation}\label{eq:2.33}
	\sum_{\text{F}}{\mathcal{P}_{\text{F}}}=1
\end{equation}
其次,我们考虑 \( \mathbf{f}_1 \cdot \mathbf{f}_2 \),由于 \( \mathbf{f}_1 \cdot \mathbf{f}_2 = \left( \mathbf{F}^2 - \mathbf{f}_1^2 - \mathbf{f}_2^2 \right)/2 \),因此 \( \mathbf{f}_1 \cdot \mathbf{f}_2 \) 仅依赖于 \( F \),可以表示为:
\begin{equation}\label{eq:2.34}
	\mathbf{f}_1\cdot \mathbf{f}_2=\sum_{\text{F}}{\left( \frac{F\left( F+1 \right)}{2}-f\left( f+1 \right) \right)}\mathcal{P}_{\text{F}}
\end{equation}
因此,我们得到
\begin{equation}\label{eq:2.35}
	\mathcal{P}_0+\mathcal{P}_2=1
\end{equation}
\begin{equation}\label{eq:2.36}
	-2\mathcal{P}_0+\mathcal{P}_2=\mathbf{f}_1\cdot \mathbf{f}_2
\end{equation}
通过求解这两个方程,可以将 $\mathcal{P}_0$ 和 $\mathcal{P}_2$ 表示为单位算符和 \( \mathbf{f}_1 \cdot \mathbf{f}_2 \) 的函数,得到
\begin{equation}\label{eq:2.37}
	\hat{V}\left( \mathbf{r} \right) =\frac{2\pi \hbar^2}{\bar{m}}\left( a^{\left( \text{n} \right)}+a^{\left( \text{s} \right)}\mathbf{f}_1\cdot \mathbf{f}_2 \right) \delta \left( \mathbf{r} \right) \partial _r\cdot r
\end{equation}
其中
\begin{equation}\label{eq:2.38}
	a^{(\mathrm{n})}=\frac{a_0+2a_2}3
\end{equation}
\begin{equation}\label{eq:2.39}
	a^{(\mathrm{s})}=\frac{a_2-a_0}3
\end{equation}
这里 $ a^{\left( \text{n} \right)} $ 和 $ a^{\left( \text{s} \right)} $ 分别表示密度–密度相互作用和自旋依赖相互作用,后者与 \( F = 0 \) 和 \( F = 2 \) 通道之间的散射长度差成正比。当 \( a_0 = a_2 \) 时,不同自旋通道的相互作用是相同的,因此 \( a^{\left( \text{s} \right)} \) 变为零。在这种情况下,相互作用仅依赖于总密度,该总密度在所有三个自旋分量的任意 $ SU\left( 3 \right)  $ 旋转下是不变的。因此,哈密顿量是 $ SU\left( 3 \right)  $ 不变的,而不是 $ SU\left( 2 \right)  $ 不变的。

这里我们还需要强调,必须仔细区分 $ SU\left( 2 \right)  $ 对称性的高自旋表示和 $ SU\left( N \right)  $ 对称性的基本表示。对于 $ SU\left( 2 \right)  $ 对称性,无论自旋 \( S \) 多大,只有三个生成元,并且相互作用在这三个生成元所产生的旋转下是不变的。在自旋 \( S \) 表象中,这三个生成元由 \( (2S+1) \times (2S+1) \) 的泡利矩阵表示。但对于 $ SU\left( N \right)  $ 对称性,总共有 \( N^2 - 1 \) 个生成元,相互作用在所有这些 \( N^2 - 1 \) 个生成元所产生的旋转下是不变的。

实际上,对于像 \( ^{87}\text{Rb} \) 和 \( ^{23}\text{Na} \) 这样的原子,\( a_0 \) 和 \( a_2 \) 之间的差异实际上是相当小的,因此 \( a^{\left( \text{s} \right)} \) 仅为 \( a^{\left( \text{n} \right)} \) 的百分之几。尽管如此,\( a^{\left( \text{s} \right)} \) 在自旋-1 碱金属原子中扮演着重要角色。从动力学角度来看,这种自旋依赖相互作用可以产生一个自旋交换过程。因为在 \( \mathbf{f}_1 \cdot \mathbf{f}_2 \) 中存在 \( f_{1+} f_{2-} + f_{1-} f_{2+} \) 项,所以两个 \( f_z = 0 \) 的原子可以散射成一个 \( f_z = 1 \) 的态和一个 \( f_z = -1 \) 的态。
\begin{tcolorbox}[colback=red!3, colframe=red!60!black, title=注释, breakable]
	\centering
    \includegraphics[width=15cm]{images/chap2-5.png}
\end{tcolorbox}


\subsection{Feshbach Resonance}\label{sec:Feshbach Resonance}
第 \ref{sec:Scattering Length} 节的讨论已经建立了一种观念,那就是散射长度是描述原子间相互作用的重要物理量。我们能否在实验上调节散射长度呢?在第 \ref{sec:Scattering Length} 节中,以方势阱为例,我们可以通过改变方势阱的深度来改变散射长度。实际上,在较大的能量范围内调节范德华势的强度是很困难的。然而,第 \ref{sec:Scattering Length} 节的讨论给了我们一个提示,即如果能够将束缚态的能量调节到接近\CJKunderdot{\textit{散射阈值}}(scattering threshold),就能使散射长度产生显著变化。这就是所有可调谐散射共振背后的主要思想。在这里,我们首先讨论磁场调谐的\CJKunderdot{\textit{费什巴赫共振}}(Feshbach resonance)。

\begin{tcolorbox}[colback=red!3, colframe=red!60!black, title=注释, breakable]
	如果要想发生散射共振,就想办法把一个束缚态放到零能去。
	散射态过来以后,它不会立刻走开,它会有很长的时间待在零能束缚态上面,一直在那里相互作用。
\end{tcolorbox}

关于磁费什巴赫共振的讨论涉及到磁场中原子的内部自旋结构。我们回想一下,在第 \ref{sec:Spin-Dependent Interaction} 节中,我们讨论了内部自旋结构对两体碰撞的影响。这里的讨论与第 \ref{sec:Spin-Dependent Interaction} 节的区别在于,我们考虑的是外部塞曼场效应足够强的情况。第 \ref{sec:Spin-Dependent Interaction} 节中我们侧重于自旋旋转对称性起重要作用的零场区域,并讨论了自旋旋转对称性如何对相互作用的形式加以限制。但在这里,有限大小的塞曼场打破了自旋旋转对称性,因此这种限制不再存在。

现在让我们更具体一些。我们考虑两个碱金属原子之间的相互作用,其在塞曼场中的内部自旋结构已在第 \ref{sec:Magnetic Structure} 节中讨论过。我们将单个原子在塞曼磁场中的每个内部自旋本征态标记为 \( |q\rangle \),它具有量子数 \( F_z \)。例如,对于 \( ^6\text{Li} \),内部自旋本征态如图 \ref{fig:2.5}(b) 所示。尽管由于塞曼磁场的存在,没有 $ SU(2) $ 的自旋旋转对称性,但沿着磁场方向仍然存在自旋旋转对称性,因此 \( F_z \) 仍然是一个好量子数。当两个原子相距较远时,它们处于 \( |q_1\rangle \otimes |q_2\rangle \) 的本征态。现在我们引入两个散射通道。一个叫做\CJKunderdot{\textit{开通道}}(open channel),另一个叫做\CJKunderdot{\textit{闭通道}}(closed channel)。这些通道定义为当两个原子相距较远时的本征态。以下是关于这两个通道的一些说明:

\begin{figure}[h!]
	\centering
	\includegraphics[width=14cm]{images/2-5.jpg}  % 指定相对路径
	\caption{两个碱金属原子之间的相互作用通道。(a) Li\(_2\) 在短程时的电子自旋单态和三重态的相互作用势。插图显示了两个 \( ^6\text{Li} \) 原子在远距离时的相互作用势的局部放大图。五对态的总 \( F_z = 0 \)。水平线表示闭通道中的束缚态。(b) \( ^6\text{Li} \) 的内部本征态,用 \( a \) 到 \( f \) 来标记。每条曲线上的数字表示每个态的 \( F_z \) 值。}
	\label{fig:2.5}
\end{figure}

\textbullet\
\CJKunderdot{\textit{量子数}}(Quantum Number):由于沿着 \( \hat{z} \) 方向的旋转对称性,总量子数 $ F_{z}^{1}+F_{z}^{2}+L_z $ 是守恒的。这里 $ F_{z}^{i=1,2} $ 分别是这两个原子超精细自旋的 \( z \)-分量,\( L_z \) 是它们相对角动量的 \( z \)-分量。为了简化,我们只考虑开通道和闭通道中的 \( s \)-波态,这时 \( L_z = 0 \)。在这种简化下,开通道的 $ F_{z}^{1}+F_{z}^{2} $ 应该等于闭通道的 $ F_{z}^{1}+F_{z}^{2} $。例如,对于 \( ^6\text{Li} \),如果将开通道设定为 \( |a\rangle \otimes |b\rangle \),总的 \( F_z = 0 \),那么总共有五种组合具有总 \( F_z = 0 \),其他四种分别是 \( |a\rangle \otimes |d\rangle \),\( |b\rangle \otimes |e\rangle \),\( |c\rangle \otimes |f\rangle \),和 \( |d\rangle \otimes |e\rangle \)。

\textbullet\
\CJKunderdot{\textit{闭通道vs开通道}}(Closed versus Open):当两个原子相距较远时,开通道与闭通道之间的能量差由塞曼能量决定,通常这个能量差远高于动能。因此,当两个原子自开通道的低能散射态发生碰撞时,它们无法散射到闭通道的散射态。这就是这些通道被称为“闭通道”的原因。如图\ref{fig:2.5}(a)的插图所示,当选择 \( |a\rangle \otimes |b\rangle \) 作为开通道时,其他四种组合都可以被视为闭通道。

\textbullet\
\CJKunderdot{\textit{能量可调控性}}(Energy Tunability):当选择低能超精细自旋态作为开通道时,随着磁场的增加,开通道的能量将相对于闭通道降低。因此可以设想,随着磁场的增加,散射阈值会从上方逐渐接近闭通道中的束缚态。


\textbullet\
\CJKunderdot{\textit{通道耦合}}(Coupling between Channels):当两个原子彼此接近时,它们之间的相互作用势主要依赖于两个原子的电子自由度。这里,由于每个碱金属原子都有一个电子,相互作用势依赖于它们的总电子自旋是单重态还是三重态(实际上,范德华相互作用对于电子自旋单态和三重态是相同的,但短程排斥相互作用则依赖于电子自旋。),图\ref{fig:2.5}(a)中给出了一个例子。例如,考虑开通道 \( |a\rangle \otimes |b\rangle \),它们的电子自旋在磁场作用下被极化,且它们的总电子自旋更接近三重态。然而,超精细耦合会使得电子自旋单态成分混合进来。因此,短程势会耦合不同的通道,尽管这种耦合通常较弱。

\textbf{Coupled-Channel Model. }在讨论了上述两个通道的特性后,我们可以考虑一个简化的耦合通道模型,来说明磁场如何改变散射长度。该模型的示意图如图 \ref{fig:2.6}(a) 所示。主要的考虑因素如下:

\begin{figure}[h!]
	\centering
	\includegraphics[width=14cm]{images/2-6.jpg}  % 指定相对路径
	\caption{费什巴赫共振。(a) 简化的两通道模型示意图。(b) 散射长度和束缚态能量的磁场依赖性,其中 \( B_0 \) 是文中提到的 \( B_{\text{res}} \)。(c)第一次从实验观察到 \( ^{23}\text{Na} \) 的费什巴赫共振。}
	\label{fig:2.6}
\end{figure}

\textbullet\
当 \( r > r_0 \) 时,两个通道解耦,分别表示为开通道 \( |o\rangle \) 和闭通道 \( |c\rangle \)。由于闭通道的能量远高于开通道中入射散射态的动能,低能散射态的波函数仅存在于开通道 \( |o\rangle \) 中,低能 \( s \)-波波函数由 \( \Psi = \chi / r \) 给出,并且
\begin{equation}
	\chi=\sin(kr+\delta_k)|o\rangle
	\label{eq:2.47}
\end{equation}
	
\textbullet\
当 \( r < r_0 \) 时,波函数在 \( |+\rangle \) 和 \( |-\rangle \) 基矢中对角化:
\begin{equation}
	\chi=\chi_+|+\rangle+\chi_-|-\rangle 
	\label{eq:2.48}
\end{equation}
其中$|+\rangle$ 和 $|-\rangle$ 是 $|o\rangle$ 和 $|c\rangle$ 的叠加:
\begin{equation}
	|+\rangle=\cos\theta|o\rangle+\sin\theta|c\rangle 
	\label{eq:2.49}
\end{equation}
\begin{equation}
	|-\rangle=-\sin\theta|o\rangle+\cos\theta|c\rangle  
	\label{eq:2.50}
\end{equation}
不失一般性,我们认为 \( \theta \) 与空间无关并且非常小。

现在,\( r < r_0 \) 区域的波函数可以写为:
\begin{equation}
	\chi=(\chi_+\sin\theta+\chi_-\cos\theta)|c\rangle+(\chi_+\cos\theta-\chi_-\sin\theta)|o\rangle   
	\label{eq:2.51}
\end{equation}

\begin{tcolorbox}[colback=red!3, colframe=red!60!black, title=注释, breakable]
	将式 \ref{eq:2.49} 和式 \ref{eq:2.50} 带入式 \ref{eq:2.48} 有
	\begin{align}
		\chi &=\chi _+\left| + \right> +\chi _-\left| - \right>  \notag \\
		&=\chi _+\left( \cos \theta \left| o \right> +\sin \theta \left| c \right> \right) +\chi _-\left( -\sin \theta \left| o \right> +\cos \theta \left| c \right> \right)  \notag \\
		&=\left( \chi _+\cos \theta -\chi _-\sin \theta \right) \left| o \right> +\left( \chi _+\sin \theta +\chi _-\cos \theta \right) \left| c \right> \tag{1}
	\end{align}
	得到式 \ref{eq:2.51}。
\end{tcolorbox}

在 \( r = r_0 \) 处匹配边界条件,我们得到:
\begin{equation}
	\chi_+\sin\theta+\chi_-\cos\theta\bigg|_{r=r_0}=0
	\label{eq:2.52}
\end{equation}
\begin{equation}
	\frac{\chi_+^{\prime}\cos\theta-\chi_-^{\prime}\sin\theta}{\chi_+\cos\theta-\chi_-\sin\theta}\bigg|_{r=r_0}^{}=\frac k{\tan\delta_k}\equiv-\frac1{a_\mathrm{s}}
	\label{eq:2.53}
\end{equation}
式 \eqref{eq:2.52} 来源于闭通道的波函数在 \( r = r_0 \) 处为零,式 \eqref{eq:2.53} 决定了开通道散射波函数中的相移。这里我们假设 \( \chi_{\pm} \) 与能量无关,且 \( r_0 \) 的值很小。上面两个等式给出:
\begin{equation}
	-\left.\frac{1}{a_{\mathrm{s}}}=\frac{\chi_{+}'}{\chi_{+}}\right|_{r=r_{0}}\cos^{2}\theta+\left.\frac{\chi_{-}'}{\chi_{-}}\right|_{r=r_{0}}\sin^{2}\theta 
	\label{eq:2.54}
\end{equation}

\begin{tcolorbox}[colback=red!3, colframe=red!60!black, title=注释, breakable]
	由式 \ref{eq:2.52} 得到
	\begin{align}
		\chi _+&=-\frac{\cos \theta}{\sin \theta}\chi _- \tag{1} \\
		\chi _-&=-\frac{\sin \theta}{\cos \theta}\chi _+ \tag{2}
	\end{align}
	带入式 \ref{eq:2.53} 得到
	\begin{align}
		-\frac{1}{a_{\text{s}}}&=\frac{\chi '_+\cos \theta -\chi '_-\sin \theta}{\chi _+\cos \theta -\chi _-\sin \theta} \notag \\
		&=\frac{\chi '_+\cos \theta}{\chi _+\cos \theta -\chi _-\sin \theta}-\frac{\chi '_-\sin \theta}{\chi _+\cos \theta -\chi _-\sin \theta} \notag \\
		&=\frac{\chi '_+\cos \theta}{\chi _+\cos \theta -\left( -\frac{\sin \theta}{\cos \theta}\chi _+ \right) \sin \theta}-\frac{\chi '_-\sin \theta}{\left( -\frac{\cos \theta}{\sin \theta}\chi _- \right) \cos \theta -\chi _-\sin \theta} \notag \\
		&=\frac{\chi '_+}{\chi _+}\frac{\cos \theta}{\cos \theta +\frac{\sin ^2\theta}{\cos \theta}}-\frac{\chi '_-}{\chi _-}\frac{\sin \theta}{-\frac{\cos ^2\theta}{\sin \theta}-\sin \theta} \notag \\
		&=\frac{\chi '_+}{\chi _+}\cos ^2\theta +\frac{\chi '_-}{\chi _-}\sin ^2\theta \tag{3}
	\end{align}
	即为式 \ref{eq:2.54}。
\end{tcolorbox}

由于 \( \theta \) 通常非常小,\eqref{eq:2.54}式右侧的第二项通常可以忽略。在这种情况下,散射长度仅由 \( |+\rangle \) 通道决定。我们令
\begin{equation}
	\left.\frac{\chi_+'}{\chi_+}\right|_{r=r_0}=-\frac1{a_{\mathrm{bg}}}
	\label{eq:2.55}
\end{equation}
其中 \( a_{\text{bg}} \) 被称为背景散射长度(background scattering length)。现在我们有
\begin{equation}
	- \frac{1}{a_\mathrm{s}}=-\frac{1}{a_\mathrm{bg}}\cos^2\theta+\left.\frac{\chi_-'}{\chi_-}\right|_{r=r_0}\sin^2\theta
	\label{eq:2.56}
\end{equation}
由于 \( \theta \) 很小,我们可以近似 \( \cos^2 \theta \approx 1 \) 和 \( \sin^2 \theta \approx \theta^2 \),第二项只有在 \( \chi'_{-}/\chi_{-} \big|_{r=r_0} \) 非常大时,才会产生显著的贡献。正如我们将看到的,这实际上意味着在 \( |-\rangle \) 通道中,束缚态出现在散射阈值附近。

例如,我们可以将 \( |-\rangle \) 通道视为一个方形势阱,\( V(r) = -V_0 \) (\( V_0 > 0 \), \( r < r_0 \)),如图\ref{fig:Z-2-6}(a)所示。在这种情况下,当散射态的能量与 \( V_0 \) 相比可以忽略不计时,我们有 \( \chi_{-} = \sin(q_1 r) \),且 \( q_1 = \sqrt{mV_0/\hbar^2} \),因此
\begin{equation}
	\frac{\chi_-^{\prime}}{\chi_-}\Bigg|_{r=r_0}=\frac{q_1\cos(q_1r_0)}{\sin(q_1r_0)}
	\label{eq:2.57}
\end{equation}
如果存在一个束缚态,其能量为 \( E_c \),那么束缚态的波函数为 \( \chi_{-}(r) = \sin(q_2 r) \),且 \( q_2 = \sqrt{m(V_0 + E_c)/\hbar^2} \)。在 \( \theta \) 的零阶近似下,\( |-\rangle \) 通道在 \( r = r_0 \) 处与 \( |c\rangle \) 通道相连,并且有 \( \sin(q_2 r_0) = 0 \)。当 \( E_c \) 很小时,我们可以在 \( q_2 \) 附近展开 \( q_1 \),得到
\begin{equation}
	\frac{\chi_-'}{\chi_-}\Bigg|_{r=r_0}\approx\frac{q_1}{(q_1-q_2)r_0}\approx\frac{2q_1^2}{(q_1^2-q_2^2)r_0}=-\frac{2\hbar^2q_1^2}{mr_0E_\mathrm{c}}
	\label{eq:2.58}
\end{equation}
实际上,尽管我们是通过方形势阱推导出式\eqref{eq:2.58},但它适用于一般的势能,其中 \( \chi'_{-}/\chi_{-} \big|_{r=r_0} \) 与 \( E_c \) 成反比。定义 \( \gamma = 2\hbar^2 q_1^2 \theta^2/(mr_0) \),式\eqref{eq:2.56}可以重新写为:
\begin{equation}
	\frac1{a_\mathrm{s}}=\frac1{a_\mathrm{bg}}+\frac\gamma{E_\mathrm{c}}
	\label{eq:2.59}
\end{equation}
这里需要注意的是,\( \gamma \) 依赖于 \( \theta \),即两个通道之间的耦合。式\eqref{eq:2.59}给出
\begin{equation}
	a_\mathrm{s}=a_\mathrm{bg}\left(1-\frac{\gamma a_\mathrm{bg}}{E_\mathrm{c}+\gamma a_\mathrm{bg}}\right)
	\label{eq:2.60}
\end{equation}

在磁场的作用下,开放通道和闭合通道的阈值能量的改变分别为 \( -\mu_o B \) 和 \( -\mu_c B \)。在大多数情况下,\( \mu = \mu_o - \mu_c > 0 \)。\( E_c \) 被替换为 \( E_c - \mu_c B + \mu_o B = E_c + \mu B \)。定义 $\Delta=\gamma a_\text{bg}\mu^{-1}$,$B_{\mathrm{res}}=-\mu^{-1}E_{\mathrm{c}}-\Delta$,式\eqref{eq:2.60}可以重新写为:
\begin{equation}
	a_\mathrm{s}=a_\mathrm{bg}\left(1-\frac{\Delta}{B-B_\mathrm{res}}\right)
	\label{eq:2.61}
\end{equation}
这个结果表明,通常对于 \( \mu > 0 \),当 $B$ 从下方接近 $B_{\text{res}}$ 时,散射长度 \( a_s \) 会发散到 \( +\infty \),而当 $B$ 从上方接近 $B_{\text{res}}$ 时,散射长度 \( a_s \) 会发散到 \( -\infty \),如图\ref{fig:Z-2-6}(b)所示。\( B_{\text{res}} \) 表示散射共振的磁场,它接近 \( E_c = 0 \) 的位置,由于 \( \Delta \) 的存在而发生偏移。\( \Delta \) 定义了共振的宽度。从式\eqref{eq:2.61}可以看出,当 \( B = B_{\text{res}} \) 时,\( a_s = \infty \),而当 \( B = B_{\text{res}} + \Delta \) 时,\( a_s = 0 \),后者被称为零交叉(zero crossing)。因此,\( \Delta \) 测量了共振散射的磁场和零交叉的磁场之间的距离。图\ref{fig:Z-2-6}(c)显示了首次在实验上观察到的 \( ^{23}\text{Na} \) 的费什巴赫共振。随后,费什巴赫共振在几乎所有碱金属和磁性原子中被发现,并成为了超冷原子物理中控制相互作用的最重要工具。

我们可以进一步证明在有限能量下:
\begin{equation}
	a_\mathrm{s}(E)=a_\mathrm{bg}\left(1-\frac{\mu\Delta}{\mu(B-B_\mathrm{res})-E}\right)
	\label{eq:2.62}
\end{equation}
做展开:\( -1/a_s(E) = -1/a_s + r_{\text{eff}} k^2/2 \),其中 \( E = \hbar^2 k^2/m \),可以得到有效范围:
\begin{equation}
	r_{\mathrm{eff}}=-\frac{2\hbar^2\Delta}{\mu ma_{\mathrm{bg}}(B-B_{\mathrm{res}}-\Delta)^2}\approx-\frac{2\hbar^2}{\mu\Delta ma_{\mathrm{bg}}}
	\label{eq:2.63}
\end{equation}
第二个约等号在共振附近有效。式\eqref{eq:2.63}表明有效范围 \( r_{\text{eff}} \) 依赖于 $\Delta$,即依赖于 \( \gamma \) 或 \( \theta \),这也是单散射通道模型和双通道模型之间的主要区别。在单通道模型中,可以通过微调势能在阈值处出现束缚态,这种共振也称为形状共振(shape resonance)。对于 \( s \)-波的形状共振,\( r_{\text{eff}} \) 通常可以忽略不计。但在双通道模型中,根据开放通道和闭合通道之间的耦合强度,有效范围可以在很宽的范围内调节,从非常小到非常大,而 \( r_{\text{eff}} \) 的符号取决于 \( a_{\text{bg}} \) 的符号。也就是说,只有当费什巴赫共振中的 \( \Delta \) 足够大时,\( r_{\text{eff}} \) 才足够小,双通道模型中的费什巴赫共振才等同于单通道模型中的形状共振。为了表征有效范围在简并费米气体的多体系统中的作用,引入一个无量纲的 \( k_F r_{\text{eff}} \):
\begin{equation}
	k_\mathrm{F}r_\mathrm{eff}=\frac{4E_\mathrm{F}}{\mu\Delta(k_\mathrm{F}a_\mathrm{bg})}
	\label{eq:2.64}
\end{equation}
如果 \( k_F r_{\text{eff}} \ll 1 \),我们称之为宽共振(wide resonance);如果 \( k_F r_{\text{eff}} \gg 1 \),我们称之为窄共振(narrow resonance)。对于窄共振而言,实际上,散射长度在能量范围 \( E_F \) 内变化较大;因此,多体系统不能仅通过散射长度 \( a_s \) 这一能量无关参数来描述。

\subsection{Confinement-Induced Resonance}\label{sec:Confinement-Induced Resonance}

\subsection{Efimov Effect}\label{sec:Efimov Effect}

\subsection{From Few to Many}\label{sec:From Few to Many}

\newpage
% 第三章
\section{Interaction Effects}\label{sec:Interaction Effects}

\subsection{Bose–Einstein Condensation}\label{sec:Bose-Einstein Condensation}

\subsection{Hydrodynamic Theory}\label{sec:Hydrodynamic Theory}

\subsection{Bogoliubov Theory}\label{sec:Bogoliubov Theory}

\subsection{One-Dimensional Bosons}\label{sec:One-Dimensional Bosons}

\subsection{Phase Coherence and Fragmentation}\label{sec:Phase Coherence and Fragmentation}

\newpage
% 第四章
\section{Topology and Symmetry}\label{sec:Topology and Symmetry}

\subsection{Soliton}\label{sec:Soliton}

\subsection{Vortex}\label{sec:Vortex}

\subsection{Spinor Condensate}\label{sec:Spinor Condensate}

\subsection{Topological Excitations in Spinor Condensate}\label{sec:Topological Excitations in Spinor Condensate}

\subsection{Spin-Orbit-Coupled Condensate}\label{sec:Spin-Orbit-Coupled Condensate}

\newpage
% 第五章
\section{The Fermi Liquid}\label{sec:The Fermi Liquid}

\subsection{Free Fermions}\label{sec:Free Fermions}

\subsection{Fermi Polaron}\label{sec:Fermi Polaron}

\newpage
% 第六章
\section{The Fermi Superfluid}\label{sec:The Fermi Superfluid}

\subsection{BCS Pairing}\label{sec:BCS Pairing}

\subsection{BCS-BEC Crossover}\label{sec:BCS-BEC Crossover}

\newpage
% 第七章
\section{Noninteracting Bands}\label{sec:Noninteracting Bands}

\subsection{Basic Band Theory}\label{sec:Basic Band Theory}

\subsection{Dirac Semimetal}\label{sec:Dirac Semimetal}

\subsection{Topological Band Insulator}\label{sec:Topological Band Insulator}

\subsection{Periodical-Driven Lattice}\label{sec:Periodical-Driven Lattice}

\subsection{Lattice from Cavity}\label{sec:Lattice from Cavity}

\newpage
% 第八章
\section{The Hubbard Model}\label{sec:The Hubbard Model}

\subsection{Bose-Hubbard Model}\label{sec:Bose-Hubbard Model}

\subsection{Fermi-Hubbard Model}\label{sec:Fermi-Hubbard Model}

\subsection{Thermalization and Entanglement}\label{sec:Thermalization and Entanglement}



\newpage
% 参考文献
\begin{thebibliography}{99}
    \bibitem{bib1} Zhai H. Ultracold atomic physics[M]. Cambridge University Press, 2021.
\end{thebibliography}

\end{document}